\chapter{Geometry}
\label{chapter:geometry}
MAUS uses the online Configuration Database to store all of its geometries. These geometries have been transferred from CAD drawings which are
modelling using the latest surveys and technical drawings available. The following section shall describe how to use the available executables to
access and use these models. 

\section{Geometry Download}

There are two executable files available to users both can be found in the directory /bin/utilities found within your copy of MAUS. The two files of
interest are download\_geometry.py and get\_geometry\_ids.py. These files do the following.

\begin{description}
  \item[Upload Geometry] \hfill 
  \begin{enumerate}
   \item Set up the Configreader class and read the values provided by ConfigurationDefaults.py or by custom config files.
   \item Instantiate an Uploader class object using the upload directory and geometry note taken from the configuration file.
   \item The list of files which is created by the Uploader class is used to compress the geometry files into one zip file.
   \item This zip file is then used as the argument for the upload\_to\_CDB method which takes the contents of the zip and then uploads this, as a
single string to the CDB.
   \item[Optional] If cleanup is specified in the configuration file then the file list and the original GDML files are the deleted leaving only the
zip file.
  \end{enumerate}

  \item[Download Geometry] \hfill
  \begin{enumerate}
   \item Set up the Configreader() class and read the values provided by ConfigurationDefaults.py or by custom config files.
   \item Instantiate a Downloader class object and downloads either the current, time specified or run number zipped geometry to a temporary cache
location.
   \item The zip file is then unzipped in this location.
   \item The Formatter class is then called and this class formats the GDMLs. The formatting alters the schema location of these files and points
them to the correct local locations of the Materials GDML file. This formatting leaves the original GDMLs in the tmeporary cache and places the new
formatted files in the download directory specified in the configuration file.
   \item GDMLtoMAUS is then called with the location of the new formatted files as its argument. This class converts the GDMLs to the MICE Module
text files using the XSLT stylesheets previously described.
   \item[Optional] If specified in the configuration file the temporary cache location is removed along with the zip file and unzipped files.
  \end{enumerate}

  \item[Get Geometry IDs]
  \begin{enumerate}
   \item Set up the Configreader() class and read the values provided by ConfigurationDefaults.py or by custom config files. This file takes start
and stop time arguments to specify a period to search the CDB.
   \item A CDB class object is then instantiated with the server specified in the configuration file.
   \item The get ids method from the CDB class is called and the python dict which is downloaded is formatted and either printed to screen or to file
as specified in the configuration file.
  \end{enumerate}
\end{description}

To use these files the user must use the arguments in the ConfigurationDefaults.py file. The arguments relating to these executables are as follows.

\begin{table*}
\begin{center}
\caption{Geometry control parameters.}
\begin{tabularx}{\textwidth}{lX}
\hline
\multicolumn{2}{l}{\emph{Geometry controls.}} \\
\hline
\verb|cdb_upload_url| & Sets the upload url relating the the Configuration Database.\\
\verb|cdb_download_url| & Sets the download url relating the the Configuration Database.\\
\verb|geometry_download_wsdl| & Name of the web service used for downloads.\\
\verb|geometry_download_directory| & Set the directory where you wish the geometry to be downloaded to.\\
\verb|geometry_download_by| & This can be set to either \textit{current, id or run\_number}. Current will download the current valid geometry stored
on the CDB. ID will download the geometry for the ID specified N.B ID numbers can be found using the get geometry ids executable. Run\_number will
download the geometry along with control room information for specified run. \\ 
\verb|geometry_download_run_number| & Set the number of the run to be downloaded.\\
\verb|geometry_download_id| & Set the number of the geometry ID to be downloaded.\\
\verb|geometry_download_cleanup| & Set to True in order to cleanup the temporary files creaated during the download process. These are the zip file
downloaded and the original GDML files from this zip file.\\
\verb|g4_step_max| & Set the G4 step max number which will be set in the ParentGeometryFile. This relates to the size of steps carried out during
the simulation.\\
\verb|geometry_upload_wsdl| & Name of the web service used for uploads. For developers use only.\\
\verb|geometry_upload_directory| & Set the the directory which stores the FastRad produced GDML files which will be stored on the CDB. For Developers
use only.\\
\verb|geometry_upload_note| & Write the description of the geometry which is going to be uploaded. This should describe what is in the beam line
specifically what is new to the model. It should also include any other information the developer wishes the user to know. For developers use only.\\
\verb|geometry_upload_valid_from| & Set the date-time format of the date when this geometry about to be uploaded is valid from. For developers use
only.\\
\verb|geometry_upload_cleanup| & Set to True in order to cleanup the temporary files creaated during the upload process. These are the file
containing the list of GDMLs to be uploaded and also the original GDML files. For developers use only.\\
\verb|get_ids_start_time| & Set the start time of the period which you would like to get the ids from the configuration database. Must be in
date-time format.\\
\verb|get_ids_stop_time| & Set the stop time of the period which you would like to get the ids from the configuration database. Must be in
date-time format.\\
\verb|get_ids_create_file| & Set to True in order to create a file which lists the geometries available within the time period specified. If set to
False the geometry information will be printed to screen.\\
\begin{makeimage} % force latex2html to render as an html table 
\end{makeimage} 
\end{tabularx}
\end{center}
\end{table*}
