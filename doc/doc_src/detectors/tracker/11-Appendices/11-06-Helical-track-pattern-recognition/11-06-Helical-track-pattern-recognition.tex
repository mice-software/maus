\section{Helical Track Pattern Recognition}
\label{App:SciFiHelicalTrackPatternRecognition}
The equation of motion of a charged particle in an external magnetic field
can be written as

\begin{equation}
\frac{d^2 x}{ds^2}=\frac{q}{p}(\frac{d \vec{x}}{ds})\times \vec{B}(s)
\end{equation}
If we assume that the magnetic filed lies along the $z$ axis $\vec{B}=(0,0,B)$
, then the three scaler components of it can be wirttien as
\begin{equation}
\begin{split}
\frac{d^2 x}{ds^2} &=\frac{q}{P}(\frac{dy}{ds})B\\
\frac{d^2 y}{ds^2} &=-\frac{q}{P}(\frac{dx}{ds})B\\
\frac{d^2 z}{ds^2} &=0
\end{split}
\end{equation}

we also note that $P$ is the total momentum and the transverse
momenumte $p_t =P \cos \lambda=qBR_H$
can be written as

\begin{equation}
\begin{split}
p_x &=p_t \cos \phi\\
p_x &=-p_t \sin \phi
\end{split}
\end{equation}

the solution of the above equations will be a helix

\begin{equation}
\begin{split}
x(s) &=x_{1} + R \left[\cos \left(\Phi_{0}+\frac{hs\cos \lambda}{R} \right)-\cos \Phi_{0} \right]\\
y(s) &=y_{1} + R \left[\cos \left(\Phi_{0}+\frac{hs\cos \lambda}{R} \right)-\sin \Phi_{0} \right]\\
z(s) &=z_{1}+s \sin \lambda
\end{split}
\end{equation}

where $x_{1}$, $y_{1}$ and $z_{1}$ is the starting point. $R$ is the radius of the helix.
$h=\pm 1$ is the sense of the rotation in $x-y$ plane. We note that
\begin{equation}
 \begin{split}
ds^2 =dx^2+dy^2+dz^2\\
ds/dz =(1+\acute{x}^2+\acute{y}^2)^{1/2}\\
(\frac{dx}{ds})^2 +(\frac{dy}{ds})^2 + (\frac{dz}{ds})^2 = 1
\end{split}
\end{equation}

On the other hand, the equation of a circle passing through three space points $(x_i,y_i)$ , where
$i=1,2,3$ can be found from the following determinant.

\begin{equation}
\left|
\begin{matrix}
x^2+y^2 & x & y & 1\\
x_1^2+y_1^2 & x_1 & y_1 & 1\\
x_2^2+y_2^2 & x_2 & y_2 & 1\\
x_3^2+y_3^3 & x_3 & y_3  & 1
\end{matrix}
\right|
=0
\end{equation}
which can be re-written as
\begin{equation}
(x^2+y^2)
\left|
\begin{matrix}
 x_1 & y_1 & 1\\
 x_2 & y_2 & 1\\
 x_3 & y_3 & 1
\end{matrix}
\right|
-x
\left|
\begin{matrix}
x_1^2+y_1^2 & y_1 & 1\\
x_2^2+y_2^2 & y_2 & 1\\
x_3^2+y_3^3 & y_3  & 1
\end{matrix}
\right|
+y
\left|
\begin{matrix}
x_1^2+y_1^2 & x_1 & 1\\
x_2^2+y_2^2 & x_2 & 1\\
x_3^2+y_3^3 & x_3 & 1
\end{matrix}
\right|
-
\left|
\begin{matrix}
x_1^2+y_1^2 & x_1 & y_1 \\
x_2^2+y_2^2 & x_2 & y_2 \\
x_3^2+y_3^3 & x_3 & y_3
\end{matrix}
\right|
=0
\end{equation}

comparing the above relation with the conventional circle equation
\begin{equation}
a(x^2+y^2)+dx+ey+f=0
\end{equation}
or
\begin{equation}
(x+\frac{d}{2a})^2+(y+\frac{e}{2a})^2=\left( \sqrt{\frac{d^2+e^2}{4a^2}-\frac{f}{a} } \right)^2
\end{equation}
we find that


\begin{equation}
a=
\left|
\begin{matrix}
 x_1 & y_1 & 1\\
 x_2 & y_2 & 1\\
 x_3 & y_3 & 1
\end{matrix}
\right|
\end{equation}

\begin{equation}
d=-
\left|
\begin{matrix}
x_1^2+y_1^2 & y_1 & 1\\
x_2^2+y_2^2 & y_2 & 1\\
x_3^2+y_3^3 & y_3  & 1
\end{matrix}
\right|
\end{equation}
\begin{equation}
e=
\left|
\begin{matrix}
x_1^2+y_1^2 & x_1 & 1\\
x_2^2+y_2^2 & x_2 & 1\\
x_3^2+y_3^3 & x_3 & 1
\end{matrix}
\right|
\end{equation}
\begin{equation}
f=-
\left|
\begin{matrix}
x_1^2+y_1^2 & x_1 & y_1 \\
x_2^2+y_2^2 & x_2 & y_2 \\
x_3^2+y_3^3 & x_3 & y_3
\end{matrix}
\right|
\end{equation}

and also the centre and the radius of the circle will be
\begin{equation}
 \begin{split}
x_0 &=-\frac{d}{2a}\\
y_0 &=-\frac{e}{2a}\\
R &=\sqrt{\frac{d^2+e^2}{4a^2}-\frac{f}{a}}
\end{split}
\end{equation}