\section{Kuno's Conjecture}
\label{App1Kuno}

For a given triplet space-point, the sum of the channel number of each
cluster will be a constant. 

To see how this comes about, consider the coordinate system defined by
the $u$, $v$ and $w$ axes in the station reference frame shown in
figure \ref{Fig:Stnuvw}.
The $u$, $v$ and $w$ coordinates my be written in terms of the polar
coordinates $(r, \phi)$ as follows:
\begin{eqnarray}
  u & = & r \cos [ \phi ]                                       \\
  v & = & r \cos \left[ \frac{2\pi}{3}-\phi \right]             \\
  w & = & r \cos \left[ \frac{4\pi}{3}-\phi \right]
\end{eqnarray}
The sum $u+v+w$ may now be written: 
\begin{eqnarray}
  u + v + w & = & r \left\{
                      \cos \phi  + 
                      \cos \left[ \frac{2\pi}{3}-\phi \right] + 
                      \cos \left[ \frac{4\pi}{3}-\phi \right] 
                    \right\}                                     \\
            & = & r \left\{
                      \cos \phi + 
                      \left[ 
                        \cos \left( \frac{2 \pi}{3} \right) \cos \phi +
                        \sin \left( \frac{2 \pi}{3} \right) \sin \phi +
                      \right] +        \right.                   \\
            &  &  ~ \left. \left[ 
                        \cos \left(-\frac{2 \pi}{3} \right) \cos \phi +
                        \sin \left(-\frac{2 \pi}{3} \right) \sin \phi +
                      \right] 
                    \right\}                                    \\
            & = & r \left\{
                      \cos \phi + 
                      2 \cos \left( \frac{2 \pi}{3} \right) \cos \phi
                    \right\}                                     \\
            & = & r \left\{
                      \cos \phi +
                      \left[ - \cos \phi \right]
                    \right\}                                     \\
            & = & 0
\end{eqnarray}
\begin{figure}
  \begin{center}
    \includegraphics[width=0.45\textwidth]%
      {11-Appendix-1-Kuno-conjecture/Figures/kuno.eps}
  \end{center}
  \caption{
    Schematic representation of a point and the three plane 
    orientations.
  }
  \label{Fig:Stnuvw}
\end{figure}

If the sum is performed using the fibre numbers for the channels hit,
the sum of the the three views will equal the sum of the
central-fibre numbers, i.e. if the central fibre numbers of each of
the $u$, $v$ and $w$ doublet-layers is $106.5$, then the sum of
channel numbers will be $106.5+106.5+105.5 = 318.5$. 
