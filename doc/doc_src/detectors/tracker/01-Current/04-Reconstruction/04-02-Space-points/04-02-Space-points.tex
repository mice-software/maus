\subsection{Space-point reconstruction}
\label{Sect:SpcPnt}

This section describes the space-point reconstruction, the algebra by
which the cluster positions are translated in to tracker coordinates
and, to some extent, the algorithm.

\subsubsection{Selection of clusters that form the space-point}

For each particle event, the clusters found within each doublet layer
are ordered by fibre-channel number.
Taking each station in turn, an attempt is made to generate a space
point using all possible combinations of clusters.
ThetThree clusters, one each from views $u$, $v$ and $w$, that make up
a space point satisfy:
\begin{equation}
 n^u + n^v + n^w = n^u_0 + n^v_0 + n^w_0 \, ;
\end{equation}
where $n^u$, $n^v$ and $n^w$ are the fibre numbers of the clusters in
the $u$, $v$ and $w$ views respectively and $n^u_0$, $n^v_0$ and
$n^w_0$ are the respective central-fibre numbers (see Appendix
\ref{App1Kuno}).
A triplet space point is selected if:
\begin{equation}
  | (n^u + n^v + n^w) - (n^u_0 + n^v_0 + n^w_0) | < K \, .
\end{equation}
Once all triplet space-points have been found, doublet space-points
are created from pairs of clusters from different views.

\subsubsection{Crossing-position calculation}

\paragraph{Doublet space-points}
\label{Para:DblSpPnt}

The position of the doublet space-point in station coordinates, 
${\bf r}_s$, is given by:
\begin{eqnarray}
  {\bf r}_s & = & \left( 
                      \begin{array}{c}
                        x_s \\ y_s
                      \end{array}
                     \right)                                  \\
               & = & \underline{\underline{R}}_{SD1} {\bf m}_1 
                                              \label{Eq:DSP1} \\
               & = & \underline{\underline{R}}_{SD2} {\bf m}_2 
                                              \label{Eq:DSP2} \, ;
\end{eqnarray}
where the measurement vector corresponding to the $i^{\rm th}$
cluster: 
\begin{equation}
  {\bf m}_i = \left( 
               \begin{array}{c}
                 \alpha_i \\ \beta_i
               \end{array}
              \right) \, ;
\end{equation}
and the rotation matrix $\underline{\underline{R}}_{SDi}$ are defined
in section \ref{HtsClstrs}.
The simultaneous equations \ref{Eq:DSP1} and \ref{Eq:DSP2} contain two
unkowns, $\beta_1$ and $\beta_2$.
Equations \ref{Eq:DSP1} and \ref{Eq:DSP2} may be rewritten:
\begin{equation}
  {\bf m}_1 = \underline{\underline{R}}_{SD1}^{-1}
              \underline{\underline{R}}_{SD2} {\bf m}_2 \, .
\end{equation}
Defining:
\begin{eqnarray}
  \underline{\underline{S}} & = & \underline{\underline{R}}_{SD1}^{-1}
                                  \underline{\underline{R}}_{SD2}     \\
                            & = & \left( 
                                    \begin{array}{cc}
                                      s_{11} & s_{12} \\
                                      s_{21} & s_{22}
                                    \end{array}
                                   \right) \, ;
\end{eqnarray}
equations \ref{Eq:DSP1} and \ref{Eq:DSP2} may be solved to yield:
\begin{eqnarray}
  \beta_2 & = & \frac{\alpha_1 - s_{11} \alpha_2}{s_{12}}     \\
  \beta_1 & = & s_{21} \alpha_2 + s_{22} \beta_2 \, .
\end{eqnarray}
The position of the space-point may now be obtained from equation
\ref{Eq:DSP1} or \ref{Eq:DSP2}.

\paragraph{Triplet space-points}

As shown in figure \ref{Fig:SenseArea}, the fibres layout is of one of
two types.
In one case (right panel of figure \ref{Fig:SenseArea}), the centre of
the channels, one in each of the three views, cross intersect at a
single point. 
In this case, the position of the crossing can be calculated as
described in seection \ref{Para:DblSpPnt}.
When the area of overlap of the three channels forms a triangle
(figure \ref{Fig:SenseArea} left panel), the centre of area of overlap
is given by:
\begin{eqnarray}
  \bar{x} & = & \frac{2}{3}c_p \, {\rm ; and}       \\
  \bar{y} & = & 0 \, .
\end{eqnarray}
