\subsection{Hits and clusters}
\label{HtsClstrs}

A track passing through a particular doublet layer produces
scintillation light in one or at most two fibre channels.
For each channel ``hit'', the tracker data aquisition system records
the channel number, $n$, and the pulse height.
After calibration, the pulse height is recorded in terms of the number
of photo-electrons ($n_{\rm pe}$) generated in the Visible Light
Photon Counter (VLPC) illuminated by the hit channel.
Occassionally, showers of particles or noise can cause three or more
neighbouring channels to be hit. 
The term ``clusters'' is used to refer to an isolated hit, a doublet
cluster and a multi-hit cluster.

The position of a hit in the doublet-layer coordinate system may be
determined from the channel number. 
For isolated hits, the measured coordinate $\alpha \in {u, v, w}$ is
given by:
\begin{equation}
  \alpha = c_p (n - n_0)\,;
\end{equation}
where $n_0$ is the channel number of the central fibre and $c_p$ is
the channel pitch given by:
\begin{equation}
  c_p = 3f_p + f_d
\end{equation}
where $f_d$ is the fibre diameter ($f_d = 350\,\mu{\rm m}$) and $f_p =
$ is the fibre pitch ($f_p = 427\,\mu{\rm m}$ see figure
\ref{Fig:DblLyr}).
For clusters in which two channels are hit (``doublet clusters'', see
figure \ref{Fig:Clust}), the measured coordinate is given by:
\begin{equation}
  \alpha = c_p \left[ \frac{( n_1 + n_2)}{2} - n_0 \right]\,;
\end{equation}
where $n_1$ and $n_2$ are the channel numbers of the two hit fibres.
For a multi-hit cluster (clusters with more than two neighbouring
channels), the measured position is determined from the pulse-height
weighted mean of the fibre positions:
\begin{equation}
  \alpha = c_p \left[ 
                 \frac{\sum_i n_{{\rm pe}i}n_i}{\sum_i n_{{\rm pe}i}} 
               \right]\,;
\end{equation}
where the subscript $i$ indicates the $i^{\rm th}$ channel.
The pulse-height for doublet and multi-channel clusters is determined
by summing the pulse height of all the hits that make up the cluster.
\begin{figure}
  \begin{center}
    \includegraphics[width=0.9\linewidth]%
    {04-Reconstruction/04-01-Hits-and-clusters/Figures/clusterRES.eps}
  \end{center}
  \caption{
    Channel overlap as simulated in G4MICE; fine-tuning reduces the
    error associated to doublet clusters.
  } 
  \label{Fig:Clust}
\end{figure}

The ``measurement vector'', ${\bf m}$ is defined as:
\begin{equation}
  {\bf m} =  \left( 
               \begin{array}{c}
                 \alpha \\ \beta
               \end{array}
              \right) \, ;
\end{equation}
where $\alpha$ is given above and, in the absense of additional
information, $\beta = 0$.
The corresponding covariance matrix is given by:
\begin{equation}
  \underline{\underline{V_m}} = 
    \left( 
      \begin{array}{cc}
         \sigma^2_\alpha & 0         \\
         0          & \sigma^2_\beta \\
       \end{array}
     \right) \, ;
\end{equation}
where $\sigma^2_\alpha$ and $\sigma^2_\beta$ are the variance of
$\alpha$ and $\beta$ respectively.
The variance on $\alpha$ for a single-hit cluster is given by:
\begin{equation}
  \sigma^2_m = \frac{c^2_p}{12} \, .
\end{equation}
For a doublet-cluster, the variance is given by:
\begin{equation}
  \sigma^2_m = \frac{\Delta^2_\alpha}{12} \, ;
\end{equation}
where $\Delta_\alpha = ?$ is the length of the overlap region between
neighbouring fibre channels (see figure \ref{Fig:Clust}).
For multihit clusters, the variance is given by
\begin{equation}
  \sigma^2_m = \frac{??}{??} \, .
\end{equation}
The variance of the perpendicular coordinate, $\beta$, depends on the
effective length, $l_{\rm eff}$ of the fibre (see figure ?? and
Appendix ??) and is given by:
\begin{equation}
  \sigma^2_\beta = \frac{l^2_{\rm eff}}{12} \, ;
\end{equation}
where
\begin{equation}
  l_{\rm eff} = ?? \; .
\end{equation}
