\subsection{Pattern recognition}
\label{Sect:PatRec}

\subsubsection{Straight-line pattern recognition}
\label{SubSect:SLPatRec}

In the absense of a magnetic field, the tracks passing through the
tracker may be described using a straight line in three dimensions.
Taking the $z$ coordinate as the independent parameter, the track
parameters may be taken to be:
\begin{equation}
 {\bf v_{\rm sl}} =
 \left( 
   \begin{array}{c}
     x_0 \\
     y_0 \\
     t_x \\
     t_y
   \end{array}
 \right)\,;
\end{equation}
where, $x_0$ and $y_0$ are the position at which the track crosses the
tracker reference surface, $t_x = \frac{dx}{dz}$ and 
$t_y = \frac{dy}{dz}$. 
The track model may then be written:
\begin{eqnarray}
  x & = & x_0 + z t_x {\rm~and} \\
  y & = & y_0 + z t_y \,.
\end{eqnarray}

Pattern recognition then procedes as follows.
A space-point is chosen in each of two stations, $i$ and $j$ where $i$
and $j$ label two different stations and $j>i$.
Ideally, $i=1$ and $j=5$.
However, a search of all combinations of pairs for which $j-i>1$ is
made, taking the pairs in the order of decreasing separation in $z$;
i.e. in order of decreasing $\Delta z_{ji} = z_j - z_i$. 
Initial values for the track parameters,
\begin{equation}
 {\bf v_{\rm sl}^{\rm Init}} =
 \left( 
   \begin{array}{c}
     x_0^{\rm Init} \\
     y_0^{\rm Init} \\
     t_x^{\rm Init} \\
     t_y^{\rm Init}
   \end{array}
 \right)\,,
\end{equation}
are then calculated as follows:
\begin{eqnarray}
  t_x^{\rm Init} & = & \frac{x_j - x_i}{z_j - z_i} \,;        \\
  x_0^{\rm Init} & = & x_i - z_i t_x^{\rm Init}      \,;        \\
  t_y^{\rm Init} & = & \frac{y_j - y_i}{z_j - z_i} \,; {\rm~and} \\
  y_0^{\rm Init} & = & y_i - z_i t_y^{\rm Init} \,;
\end{eqnarray}
where $(x_i, y_i, z_i)$ are the coordinates of space-point $i$, etc.
A search is then made for space-points in each of the stations, $k$,
between station $i$ and station $j$.
The distance between the $x$ and $y$ coordinates of the space-points
in the stations $k;\,j<k<i$ and the line defined by the initial track
parameters is then calculated at the reference surface of station $k$
as follows:
\begin{eqnarray}
  \delta x_k & = & x_k - (x_0^{\rm Init} + z_k t_x^{\rm Init}) {\rm~and} \\
  \delta y_k & = & y_k - (y_0^{\rm Init} + z_k t_y^{\rm Init}) \,.
\end{eqnarray}
Points are accepted as part of a ``trial'' track if:
\begin{eqnarray}
  | \delta x_k | & < & \Delta_x {\rm~and} \\
  | \delta y_k | & < & \Delta_y \,.
\end{eqnarray}
If at least one space-point statisfies this selection, a ``trial track''
is formed consisting of the space-points selected in stations 
$i$, $k$, ... and $j$.

For each ``trial track'', a straight-line fit is performed to
calculate the fit $\chi^2$:
\begin{equation}
  \chi^2 = \chi_x^2 + \chi_y^2 \,.
\end{equation}
Expressions for the evaluation of $\chi^2_x$ and $\chi^2_y$ are given
in Appendix \ref{App:LS-SL-fit}.
If the fit $\chi^2$ satisfies:
\begin{equation}
  \frac{\chi^2}{N-2} < \chi^2_{\rm cut} \,,
\end{equation}
then the trial track is accepted and the pattern-recognition track
parameters and covariance matrix are calculated using the expressions
given in Appendix \ref{App:LS-SL-fit}.

\subsubsection{Helix pattern recognition}
\label{SubSect:SLPatRec}

\paragraph{Helix parameters}

In the presence of a magnetic field, the tracks passing through the
tracker may be described using a helix.
In tracker coordinates, the tracks form circles in the $(x, y)$
plane.
Defining $s$ to be the length of the arc swept out by the track in the
$(x, y)$ plane, a track may be described using a straight line in the
$(s, z)$ plane.
Taking the $z$ coordinate as the independent parameter, the track
parameters may be taken to be:
\begin{equation}
 {\bf v_{\rm hlx}} =
 \left( 
   \begin{array}{c}
     x_0    \\
     y_0    \\
     \psi_0 \\
     t_s    \\
     \rho
   \end{array}
 \right)\,;
\end{equation}
where, $x_0$ and $y_0$ are the position at which the track crosses the
tracker reference surface, $\psi$ is the azimuthal angle of the line
tangent to the track in the $(x, y)$ plane, $t_s = \frac{ds}{dz}$ and 
$\rho$ is the radius of curvature. 
The angle $\psi$ is chosen such that:
\begin{equation}
  {\bf \hat{\psi}_0} = {\bf \hat{k}} \times {\bf \hat{r}} \, ;
\end{equation}
where ${\bf \hat{r}}$ is the unit vector in the direction $(x_0, y_0)$
and ${\bf \hat{k}}$ is the unit vector parallel to the $z$ axis.
${\bf \hat{\psi}_0}$ is the unit vector tangent to the track and in
the direction is defined by $\psi_0$.
With this definition, the projection on the $(x, y)$ plane of a
positive track propagating in the positive $z$ direction sweeps
anticlockwise.

\paragraph{Track model for pattern recognition}

To build up the track model, consider a track-based coordinate system
which has its origin at the point $(x_0, y_0)$ and in which the
$x'$ axis is parallel to the line joining $(x_0, y_0)$ to the
centre of the circle descibed by the track, the $y'$ axis is parallel
to ${\bf \hat{\psi}_0}$ and the $z'$ axis is parallel to 
${\bf \hat{k}}$ (see figure \ref{Fig:PatRecTrkMdl}).
In the primed coordinate system, the $(x', y')$ track model may now be
written: 
\begin{eqnarray}
  x' = \rho \
\end{eqnarray}

The track model may now be written:
\begin{eqnarray}
  x & = & x_0 + z t_x {\rm~and} \\
  y & = & y_0 + z t_y \,.
\end{eqnarray}

Pattern recognition then procedes as follows.
A space-point is chosen in each of two stations, $i$ and $j$ where $i$
and $j$ label two different stations and $j>i$.
Ideally, $i=1$ and $j=5$.
However, a search of all combinations of pairs for which $j-i>1$ is
made, taking the pairs in the order of decreasing separation in $z$;
i.e. in order of decreasing $\Delta z_{ji} = z_j - z_i$. 
Initial values for the track parameters,
\begin{equation}
 {\bf v_{\rm sl}^{\rm Init}} =
 \left( 
   \begin{array}{c}
     x_0^{\rm Init} \\
     y_0^{\rm Init} \\
     t_x^{\rm Init} \\
     t_y^{\rm Init}
   \end{array}
 \right)\,,
\end{equation}
are then calculated as follows:
\begin{eqnarray}
  t_x^{\rm Init} & = & \frac{x_j - x_i}{z_j - z_i} \,;        \\
  x_0^{\rm Init} & = & x_i - z_i t_x^{\rm Init}      \,;        \\
  t_y^{\rm Init} & = & \frac{y_j - y_i}{z_j - z_i} \,; {\rm~and} \\
  y_0^{\rm Init} & = & y_i - z_i t_y^{\rm Init} \,;
\end{eqnarray}
where $(x_i, y_i, z_i)$ are the coordinates of space-point $i$, etc.
A search is then made for space-points in each of the stations, $k$,
between station $i$ and station $j$.
The distance between the $x$ and $y$ coordinates of the space-points
in the stations $k;\,j<k<i$ and the line defined by the initial track
parameters is then calculated at the reference surface of station $k$
as follows:
\begin{eqnarray}
  \delta x_k & = & x_k - (x_0^{\rm Init} + z_k t_x^{\rm Init}) {\rm~and} \\
  \delta y_k & = & y_k - (y_0^{\rm Init} + z_k t_y^{\rm Init}) \,.
\end{eqnarray}
Points are accepted as part of a ``trial'' track if:
\begin{eqnarray}
  | \delta x_k | & < & \Delta_x {\rm~and} \\
  | \delta y_k | & < & \Delta_y \,.
\end{eqnarray}
If at least one space-point statisfies this selection, a ``trial track''
is formed consisting of the space-points selected in stations 
$i$, $k$, ... and $j$.

For each ``trial track'', a straight-line fit is performed to
calculate the fit $\chi^2$:
\begin{equation}
  \chi^2 = \chi_x^2 + \chi_y^2 \,.
\end{equation}
Expressions for the evaluation of $\chi^2_x$ and $\chi^2_y$ are given
in Appendix \ref{App:LS-SL-fit}.
If the fit $\chi^2$ satisfies:
\begin{equation}
  \frac{\chi^2}{N-2} < \chi^2_{\rm cut} \,,
\end{equation}
then the trial track is accepted and the pattern-recognition track
parameters and covariance matrix are calculated using the expressions
given in Appendix \ref{App:LS-SL-fit}.
