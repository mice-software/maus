\section{Space-point variance}
\label{App:SpcPntSD}

Figure \ref{Fig:SenseArea} shows the arrangement of the fibre channels in
the tracker.
The regions in which a space point will be reconstructed are shown by
the shaded areas.
The area of the triangular intersection is given by:
\begin{eqnarray}
  A & = & 4 \frac{1}{2} \frac{c_p}{\sqrt{3}} \frac{c_p}{2} \\
    & = & \frac{c_p^{2}}{\sqrt{3}} \, ;
\end{eqnarray}
where $c_p$ is the channel pitch.
Therefore, for the triangular intersection, the mean values of $x$ and
$y$ are given by:
\begin{eqnarray}
  \bar{x} & = & \frac{1}{A} \int \int x dx dy                           \\
          & = & \frac{1}{A} \int_{0}^{c_p} x dx                         
                \int_{-\frac{x}{\sqrt{3}}}^{\frac{x}{\sqrt{3}}} dy                \\
          & = & \frac{1}{A} \frac{2}{\sqrt{3}} \int_{0}^{c_p} x^{2} dx   \\
          & = & \frac{1}{A} \frac{2}{\sqrt{3}} \frac{c_p^{3}}{3}        \\
          & = & \frac{\sqrt{3}}{c_p^2} \frac{2}{\sqrt{3}} \frac{c_p^{3}}{3}  \\
          & = & \frac{2}{3}c_p \, {\rm ; and}
\end{eqnarray}
\begin{eqnarray}
  \bar{y}&=&\frac{1}{A} \int \int y dx dy                            \\
         &=&\frac{1}{A} \int_{0}^{c_p} dx                            
            \int_{-\frac{x}{\sqrt{3}}}^{\frac{x}{\sqrt{3}}} y dy               \\
         &=&\frac{1}{A}  \int_{0}^{c_p} 
            \left[ \frac{y^{2}}{2} \right]_{-\frac{x}{\sqrt{3}}}^{\frac{x}{\sqrt{3}}} dx \\
         &=&0 \, .
\end{eqnarray}
The variance of the $x$ and $y$ coodinates are then given by:
\begin{eqnarray}
  V_x = \sigma^2_x &=& \frac{1}{A} \int \int (x-\bar{x})^{2} dx dy   \\
                  &=& \frac{1}{A} \int_{0}^{c_p} dx 
                      \int_{-\frac{x}{\sqrt{3}}}^{\frac{x}{\sqrt{3}}}  
                      (x-\bar{x})^{2} dy                            \\
                  &=& \frac{1}{A} \frac{2}{\sqrt{3}} 
                      \int_{0}^{c_p} x (x-\bar{x})^{2}              \\
                  &=& \frac{1}{A} \frac{2}{\sqrt{3}} 
                      \int_{0}^{c_p}
                      ( x^{3}-2x^{2}\bar{x}+\bar{x}^{2}x ) dx           \\
                  &=& \frac{1}{A} \frac{2}{\sqrt{3}} 
                      \int_{0}^{c_p} ( x^{3}-\frac{4}{3}c_p x^{2}
                      + \frac{4}{9} c_p^{2} x ) dx                  \\
                  &=& \frac{1}{A} \frac{2}{\sqrt{3}} 
                      \left[ \frac{x^{4}}{4} - \frac{4}{9} c_p x^{3}
                      + \frac{2}{9} c_p^{2} x^{2}\right]_{0}^{c_p}      \\
                  &=& \frac{c_p^4}{A} \frac{2}{\sqrt{3}} 
                      \left[ \frac{1}{4}-\frac{4}{9}+\frac{2}{9} \right]  \\
                  &=& \frac{c_p^4}{A} \frac{2}{\sqrt{3}} 
                      \left[ \frac{1}{4}-\frac{2}{9} \right]  \\
                  &=& c_p^4 \frac{\sqrt{3}}{c_p^2} \frac{2}{\sqrt{3}} 
                           \frac{1}{36}  \\
                  &=& \frac{1}{18} c_p^{2} \\
  V_x = \sigma^2_x &=& \left( \frac{c_p}{3\sqrt{2}} \right)^2 \, ;
\end{eqnarray}
\begin{eqnarray}
  V_y = \sigma^2_y 
        &=& \frac{1}{A} \int_{0}^{c_p} dx
            \int_{-\frac{x}{\sqrt{3}}}^{\frac{x}{\sqrt{3}}}(y-\bar{y})^{2} dy \\
        &=& \frac{1}{A} \int_{0}^{c_p} \left[ \frac{y^{3}}{3} 
            \right]_{-\frac{x}{\sqrt{3}}}^{\frac{x}{\sqrt{3}}} dy            \\
        &=& \frac{1}{A} \frac{2}{3\sqrt{3}} 
            \int_{0}^{c_p} x^{3}dx \nonumber                        \\
        &=& \frac{1}{A} \frac{2}{9\sqrt{3}} 
            \left[ \frac{x^{4}}{4} \right]_{0}^{c_p}                \\
        &=& \frac{1}{A} \frac{2}{9\sqrt{3}} \frac{c_p^4}{4}      \\
        &=& \frac{\sqrt{3}}{c_p^2} \frac{2}{9\sqrt{3}} \frac{c_p^4}{4}  \\
        &=& \frac{1}{9} \frac{c_p^2}{2}  \\
        &=& \frac{1}{18} c_p^2  \\
  V_y = \sigma^2_y 
        &=& \left( \frac{c_p^{2}}{3\sqrt{2}} \right)^2 \, .
\end{eqnarray}
The covariance is given by:
\begin{eqnarray}
  V_{xy} &=& \frac{1}{A} \int \int (x-\bar{x}) (y-\bar{y}) dx dy         \\
        &=& \frac{1}{A} \int_{0}^{c_p} (x-\bar{x}) dx
            \int_{-\frac{x}{\sqrt{3}}}^{\frac{x}{\sqrt{3}}}(y-\bar{y}) dy     \\
        &=& \frac{1}{A} \int_{0}^{c_p} (x-\bar{x})
            \left[ 
              \frac{1}{2}y^2 - y \bar{y} 
            \right]_{-\frac{x}{\sqrt{3}}}^{\frac{x}{\sqrt{3}}}                \\
        &=& 0 \, .
\end{eqnarray}
Therefore: 
\begin{equation}
  \sigma_{x}=\sigma_{y}=\frac{c_p}{3\sqrt{2}} = 384.4 \mu{\rm m} \, .
\end{equation}
For the hexagonal case, the area of the overlapping region (shaded
zone in the right panel of figure \ref{Fig:SenseArea}) is givn by:
\begin{eqnarray}
  A & = & 6 \frac{1}{2} \frac{c_p}{\sqrt{3}} \frac{c_p}{2} \\
    & = & \frac{\sqrt{3}}{2}c_p^{2} \, .
\end{eqnarray}
By symmetry, $\bar{x} = \bar{y} = 0$.
The variance of the $x$ and $y$ coordinates are given by:
\begin{eqnarray}
  V_x = \sigma^2_x = \sigma^2_y =
        &=& \frac{1}{A} \int \int (x-\bar{x})^{2} dx dy \\
        &=& \frac{1}{A} \int \int x^{2} dx dy \\
        &=& \frac{2}{A}  \int_{-\frac{c_p}{2}}^{0}x^{2} dx 
            \int_{-\frac{x}{\sqrt{3}}-\frac{c_p}{\sqrt{3}}}
                ^{\frac{x}{\sqrt{3}}+\frac{c_p}{\sqrt{3}}} dy \nonumber \\
        &=& \frac{2}{A}\int_{-\frac{c_p}{2}}^{0}x^{2} 
            \left[ 2 \left( \frac{x}{\sqrt{3}}+\frac{c_p}{\sqrt{3}}  
            \right) \right] dx \\
        &=& \frac{2}{A}\frac{2}{\sqrt{3}} \int_{-\frac{c_p}{2}}^{0} 
            \left( x^{3}+ x^{2}c_p \right)dx \nonumber \\
        &=& \frac{2}{A}\frac{2}{\sqrt{3}} \left[ \frac{1}{4}
            x^4 +\frac{1}{3} x^3 c_p\right]_{-\frac{c_p}{2}}^{0} \\
        &=& \frac{2}{A}\frac{2}{\sqrt{3}} \left[ -\frac{1}{4}
            \frac{c_p^{4}}{16}+\frac{1}{3}\frac{c_p^{4}}{8}\right] \\
        &=& \frac{2}{A}\frac{2}{\sqrt{3}}
            \left[ \frac{1}{8} \left( \frac{1}{3}-\frac{1}{8} 
            \right) c_p^4 \right]\\
        &=& \frac{2}{A}\frac{2}{\sqrt{3}} \frac{1}{8} \frac{5}{24} c_p^4
         =  \frac{1}{A} \frac{5}{48\sqrt{3}} c_p^4 \\
        &=& \frac{2}{\sqrt{3}c_p^{2}} \frac{5}{48\sqrt{3}} c_p^4 
         =  \frac{2}{\sqrt{3}} \frac{5}{48\sqrt{3}} c_p^2 \\
        &=& \left( \sqrt{\frac{5}{2}}\frac{c_p}{6} \right)^2 \, .
\end{eqnarray}
As before, the covariance is given by:
\begin{eqnarray}
  V_{xy} &=& \frac{1}{A} \int \int (x-\bar{x}) (y-\bar{y}) dx dy         \\
        &=& \frac{2}{A}  \int_{-\frac{c_p}{2}}^{0} x dx 
            \int_{-\frac{x}{\sqrt{3}}-\frac{c_p}{\sqrt{3}}}
                ^{\frac{x}{\sqrt{3}}+\frac{c_p}{\sqrt{3}}} y dy \nonumber \\
        &=& \frac{2}{A}  \int_{-\frac{c_p}{2}}^{0} x dx 
            \left[
             \frac{1}{2} y^2
            \right]_{-\frac{x}{\sqrt{3}}-\frac{c_p}{\sqrt{3}}}
                   ^{\frac{x}{\sqrt{3}}+\frac{c_p}{\sqrt{3}}}   \nonumber \\
        &=& 0 \, .
\end{eqnarray}
Therefore:
\begin{equation}
  \sigma_{x}=\sigma_{y}=\sqrt{\frac{5}{2}}\frac{c_p}{6} = 429.8 \mu{\rm m} \, .
\end{equation}
\begin{figure}
  \begin{center}
    \includegraphics[width=0.7\linewidth]%
      {12-Appendix-2-Space-point-standard-deviation/Figures/drawing2.eps}
  \end{center}
  \caption{
      Right panel: Fibre arrangement in station 5 of tracker 1. 
      Left panel: Fibre arrangement in the rest of the stations. 
      The shaded region shows the intersection of the three channels
      is triangle for every station other than station 5, where it
      will be an hexagon.
  }
  \label{Fig:SenseArea}
\end{figure}
