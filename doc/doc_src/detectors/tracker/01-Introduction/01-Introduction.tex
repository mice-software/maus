\section{Introduction}
\label{Sect:SciFiIntroduction}

\subsection{Overview}
\label{SubSect:SciFiOverview}

This chapter describes the software used to simulate and reconstruct the MICE scintillating fibre trackers. Section~\ref{Sect:SciFiDefinitions}, \ref{Sect:SciFiRefSurf&CoordSysts} and \ref{Sect:SciFiReconstructionAlgorithms} are reference sections providing descriptions of the official definitions, reference surfaces and coordinate systems, and reconstruction algorithms respectively.  The later sections provide descriptions of the code as implemented in MAUS.  A quick start guide regular users appears in below in section~\ref{SubSect:SciFiQuickStart}.

\subsection{Quick start guide}
\label{SubSect:SciFiQuickStart}

Example scripts and datacards for the tracker reconstruction can be found in the \verb;bin/user/scifi; directory.  A typical top level python file to run a simulation with tracker reconstruction is shown below.

\begin{lstlisting}[
    language=Python,
    caption={Example SciFi python script},
    label=api:IModule, 
    index={IModule},
    emph={birth,death},
    emphstyle={\color{DarkBlue}}
  ]
import io   #  generic python library for I/O
import gzip #  For compressed output # pylint: disable=W0611
import MAUS

def run():
    # This input generates empty spills,
    # to be filled by the beam maker later
    my_input = MAUS.InputPySpillGenerator()

    # The mappers for to set up the simulation
    my_map = MAUS.MapPyGroup()
    my_map.append(MAUS.MapPyBeamMaker()) # beam construction
    my_map.append(MAUS.MapCppSimulation())  #  geant4 sim
    
    # The mappers for tracker MC digitisation and recon
    my_map.append(MAUS.MapCppTrackerMCDigitization()) 
    my_map.append(MAUS.MapCppTrackerRecon())
    
    # Specify config parameters via a datacard
    datacards = io.StringIO(u"")

    # The Pattern Recognition reducer to display tracks
    reducer = MAUS.ReduceCppPatternRecognition()

    # Output to ROOT file
    my_output = MAUS.OutputCppRoot()

    # The Go() drives all the components you pass in
    MAUS.Go(my_input, my_map, reducer, my_output, datacards)

if __name__ == '__main__':
    run()
\end{lstlisting}

\noindent
Some important datacard parameters to consider when using the tracker software are:

\begin{itemize}
 \item \verb;SciFiPRHelicalOn; - set to True or False for helical pattern recognition
 \item \verb;SciFiPRStraightOn; - set to True or False for straight pattern recognition
 \item \verb;SciFiKalmanOn; - set to True or False for running the final track fit
\end{itemize}
