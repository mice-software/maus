\chapter{Accessing Global Tracks}
\label{chapter:globalds}

The global datastructure is set up so that the state of tracks at any point in the processing chain, import,
track matching, pid, and track fitting, can easily be accessed in order to understand where a given track
came from and what lead to its current state.

To achieve this, tracks are packaged together in PrimaryChain objects which retain copies of the
tracks at every step along the way. There will generally be one primary chain for every tracker track that
has been matched to at least one additional detector, plus possible through primary chains for tracks
that have been matched through the absorber.

\section{The PrimaryChain Object}\label{sec:primarychain}

A primary chain (\texttt{MAUS::DataStructure::Global::PrimaryChain}) has a number of member functions
for accessing parameters describing the chain as well as the variety of tracks that are contained in it:

\begin{description}
  \item [get\_chain\_type()] A chain can hold either upstream tracks, downstream tracks, or through tracks. The chain type is an enumerator with the following values
  \begin{description}
    \item [kNoChainType] Used for initialization only
    \item [kUS] A chain containing upstream tracks. This value furthermore implies that a track in the chain has been matched into a through chain.
    \item [kDS] A chain containing downstream tracks. This value furthermore implies that a track in the chain has been matched into a through chain.
    \item [kUSOrphan] A chain containing upstream tracks none of which has been matched into a through track
    \item [kDSOrphan] A chain containing downstream tracks none of which has been matched into a through track
    \item [kThrough] A chain containing through tracks. It will also reference daughter upstream and downstream chains
  \end{description}
  \item [get\_chain\_multiplicity] This is only a meaningful property for through chains. It describes whether this and another through chain are mutually exclusive
        because they share daughter chains, for example if the same upstream track has been matched to multiple downstream tracks. The possible values of the enumerator are
  \begin{description}
    \item [kUnique] This chain does not share daughter chains with any other through chains
    \item [kMultipleUS] This chain shares an upstream chain with at least one other through chain
    \item [kMultipleDS] This chain shares a downstream chain with at least one other through chain
    \item [kMultipleBoth] This chain shares both its upstream and downstream chains with other through chains
  \end{description}
  Note that constellations are possible where e.g. there are 3 chains, one of type \texttt{kMultipleUS}, one of type \texttt{kMultipleDS},
  and the last of type \texttt{kMultipleBoth}.
  \item [get\_mapper\_name()] The name of the mapper that last modified this primary chain object.
  \item [GetUSDaughter()] For a through chain returns the upstream daughter chain.
  \item [GetDSDaughter()] For a through chain returns the downstream daughter chain.
  \item [GetMatchedTracks()] Returns a vector of all matched tracks for this chain. There is not a single matched track since as described in section ref{sec:tmprocess},
        tracks for all feasible PID hypotheses have to be created.
  \item [GetPIDTrack()] Returns the PID'd track (if one exists in the chain).
  \item [GetFittedTrack()] Returns the fitted track (if one exists in the chain).
\end{description}

Note that the global track object also holds references to constituent tracks, so there are two different
ways one could obtain upstream and downstream tracks from a through primary chain. For example to obtain PID'd tracks:
\begin{verbatim}
through_chain->GetPIDTrack()->GetConstituentTracks()
\end{verbatim}
or
\begin{verbatim}
through_chain->GetUSDaughter()->GetPIDTrack()
through_chain->GetDSDaughter()->GetPIDTrack()
\end{verbatim}

\subsection{Identifying Decay Candidates}\label{subsec:decay_candidates}

Primary chains also allow identifying events that may contain a $\pi \rightarrow \mu$ or $\mu \rightarrow e$ decay. If an upstream
and downstream track come out of Particle ID with different PIDs, no PID'd through track can be created, so an existing through
chain will return NULL from \texttt{through\_chain->GetPIDTrack()}. In this case, a decay candidate can be identified e.g. by testing that
\begin{verbatim}
through_chain->GetUSDaughter()->GetPIDTrack()->get_pid()
    == MAUS::DataStructure::Global::kPiPlus
\end{verbatim}
and
\begin{verbatim}
through_chain->GetDSDaughter()->GetPIDTrack()->get_pid()
    == MAUS::DataStructure::Global::kMuPlus
\end{verbatim}

Additional candidates could be identified from orphan chains, but caution is advised, as there is a good chance these might be from different particles rather than decays.

\section{Tracks and Space Points from Local Reconstruction}\label{sec:lr_tracks_spacepoints}

While tracks and space points from the local reconstruction which have ended up in matched tracks can be accessed as constituent tracks of matched tracks and
as space points linked to by the track points in matched tracks, there might be a need to access all tracks and spacepoints from local reconstruction,
including those that were not successfully matched. Since they can't be associated with a primary chain, they can be accessed direcly from the global event via
\begin{verbatim}
std::vector<MAUS::DataStructure::Global::Track*> GetLRTracks()
std::vector<MAUS::DataStructure::Global::Track*>
    GetLRTracks(MAUS::DataStructure::Global::DetectorPoint detector)
\end{verbatim}
and
\begin{verbatim}
std::vector<MAUS::DataStructure::Global::SpacePoint*> GetLRSpacePoints()
std::vector<MAUS::DataStructure::Global::SpacePoint*>
    GetLRSpacePoints(MAUS::DataStructure::Global::DetectorPoint detector)
\end{verbatim}
where the optional argument detector allows specifying from which detectors tracks or space points should be retrieved.
