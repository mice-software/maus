\chapter{Global Track Matching}
\label{chapter:globaltrackmatching}

\section{Introduction}\label{sec:tmintro}

\section{Purpose}\label{sec:tmpurpose}

The purpose of Track Matching is to assembly hits and tracks in the various detectors into global tracks,
(upstream, downstream, and through-going) by determining which hits belong together and creating new tracks
containing them. PID (see next section) can then be performed on the resulting tracks.

\section{Process}\label{sec:tmprocess}

Track Matching is performed in several steps. First, tracks for the upstream and downstream sections of the
beamline are assembled by matching hits and tracks from the various other detectors to tracks produced by the
trackers (if no tracker track exists, no matching is performed). Then, upstream and downstream tracks from the
event are matched together.

As particle masses and charges are required for propagation (see section \ref{subsec:tmrk4}), multiple tracks
are created for the various PID hypotheses. If the tracker can produce a charge hypothesis due to the direction
of the helical track, three tracks are created (in case of pion, muon, or electron of the corresponding charge),
otherwise six.

Note that in the default setting, the upstream and downstream tracks are only created by explicit matching if
one of the detectors has more than one hit (or track), otherwise all it is simply assumed that the hits come from
the same particle and tracks are assembled accordingly.

\subsection{4th Order Runge-Kutta Propagation}\label{subsec:tmrk4}

Matching to the various detectors (see below) is generally done using a 4th order Runge-Kutta propagation routine
(implemented as a wrapper for the GSL RK4 Integration). A trackpoint from the outermost tracker station (most
upstream for the upstream beamline, most downstream for the downstream beamline) provides the input which is then
propagated backwards / forwards to the other detectors. A maximum allowed disagreement between x and y position of
propagated vs. reconstructed hits (detector resolution plus an additional allowance for multiple scattering) is
typically used as matching criterion (see below).

Energy loss is included in the propagation and implemented as follows:

After every step, the midpoint between previous and current position is calculated and the material at that point
obtained from the geometry. This, together with PID, energy, and distance between previous and current positions
is used to calculate energy loss (or energy gain for backwards-propagation) via the Bethe-Bloch formula. Furthermore,
at every step, the distance to the nearest material boundary is determined, and if lower than the step size, the
step size is reduced significantly, both to minimize inaccuracies arising from an underestimated straight-line step
distance in materials with high stopping power (Material layers in the beamline are all relatively thin, so in such
materials the nearest boundary will always be close) and to ensure that the material during a step is uniform.

\subsection{TOF1, TOF2, KL}\label{subsec:tmtof12kl}

For TOF1, TOF2, and the KL, the matching tolerance is a fixed value that can be configured for each detector in the
datacards (see section \ref{sec:tmconfig}).

\subsection{TOF0}\label{subsec:tmtof0}

The beamline optics between TOF1 and TOF0 make propagation-based matching to TOF0 unfeasible (typical position and
momentum errors in the tracker are vastly magnified once propagation has reached TOF0), so a slightly different method
is used: First, propagation is performed to the upstream end of TOF1. Then, based on the momentum components at this
point, the travel distance from TOF0 is estimated and used to calculate an approximate energy loss over that distance.
By applying half of this as an energy-gain to the particle on the upstream edge of TOF1, we obtain an approximate
(as we do not account for fringe-field effects from Q7-9) average energy for the travel between the TOFs. From this,
we can obtain an expected $\Delta t$ between TOF0 and TOF1, which is then compared to the reconstructed $\Delta t$.
The maximum allowed discrepancy between the two can be configured in the datacards.

Note that TOF0 matching can only be performed if TOF1 matching was successful.

\subsection{Cherenkov Detectors}\label{subsec:tmckovs}

As the Cherenkov detectors don't have the time resolution to resolve multiple particles within a trigger window,
Cherenkov hits are added to the track without any checks.

\subsection{EMR}\label{subsec:tmemr}

Matching to the EMR (specifically to the most upstream trackpoint of an EMR track) works similar to matching to TOF1,
TOF2, and KL, but rather than using a fixed distance as a matching tolerance, the tolernace is scaled by the position
error reported by the EMR. A scaling coefficient can be configured in the datacards.

\subsection{Upstream-Downstream Matching}\label{subsec:tmusdsmatching}

For Upstream-Downstream matching, a cut is imposed on the maximum and minimum average $\beta_{z}$ (where
$\beta = \frac{v}{c}$, not the beta function) between TOF1 and TOF2. The maximum and minimum can be configured in the
datacards.

\section{Usage}\label{sec:tmusage}

To use Track Matching in a MAUS module chain, simply include the mappers \texttt{MapCppGlobalReconImport} (which
imports all locally reconstructed data into a global event) and \texttt{MapCppGlobalTrackMatching} in your mapper chain
just after all local reconstruction mappers (or as the first mappers if you are working with data that is already unpacked).

\section{Configuration}\label{sec:tmconfig}

\begin{description}
  \item[track\_matching\_pid\_hypothesis] If the PID hypotheses used for track matching should be limited to a single PID,
      change to either one (never several) of "kEPlus", "kEMinus", "kMuPlus", "kMuMinus", "kPiPlus", "kPiMinus".
  \item[track\_matching\_tolerances] Global Track Matching tolerances for the various subdetectors
  \begin{description}
    \item[TOF0t] Tolerance in ns for matching to TOF0.
    \item[TOF1x, TOF1y, TOF2x, TOF2y, KLy] Tolerance in mm. For the KL, only the y coordinate is used, as the KL does not
        resolve the x-position.
    \item[EMRx, EMRy] Multiplier for the standard tolerance for the EMR which is the reconstructed error $\times \sqrt{12}$
    \item[TOF12maxSpeed, TOF12minSpeed] Maximum and minimum average speed of the particle between TOF1 and TOF2 as a fraction
        of $c$ in order to match an upstream and a downstream track.
  \end{description}
  \item[track\_matching\_energy\_loss] Whether to use energy calculations for Global Track Matching. If for a given run
      there are no fields between TOF1 and EMR, this can be set to \texttt{False} for a slight speed-up, though the TOF0
      tolerance would have to be increased. Safer to leave on.
  \item[track\_matching\_no\_single\_event] Whether propagation matching (for individual detectors, this does not affect the
      Upstream-Downstream Matching (see section \ref{subsec:tmusdsmatching}) should not be performed if each detector has no
      more than one hit (significantly increases execution speed).
  \item[track\_matching\_charge\_thresholds] If \texttt{track\_matching\_no\_single\_event} is activated, this will cause propagation
      matching to still occur if one of the hits has a charge deposit below the threshold (i.e. is likely noise). \textbf{NOT CURRENTLY IMPLEMENTED}
\end{description}
