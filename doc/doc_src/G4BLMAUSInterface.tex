\input{rgb}

\lstset{% general command to set parameter(s)                                                                                                                                                                                                
frame=single,
basicstyle=\small,
keywordstyle=\color{DarkViolet}\bfseries,
identifierstyle=\color{DarkGreen},
commentstyle=\color{DarkRed},
stringstyle=\ttfamily,% typewriter type for strings                                                                                                                                                                                          
showstringspaces=false,% no special string spaces                                                                                                                                                                                            
captionpos=b% caption at bottom                                                                                                                                                                                                              
}

\chapter{G4beamline-MAUS Integration}

This chapter describes how to run G4beamline as a third-party app with MAUS. G4beamline is used to model the MICE beam line from the target to a point upstream of the second quad triplet (upstream of Q4). It provides a realistic beam desciption which can be used to seed downstream simulations in MAUS. To generate MAUS primaries with G4beamline the script \verb+simulate\_mice\_G4BL.py+ is used instead of the usual \verb+simulate\_mice.py+ script. The beam line settings can be controlled with the dictionary g4bl (table \ref{table}) in the MAUS datacard. 

\begin{table}
\caption{G4BL parameters}
\begin{tabular}{|l | p{10cm}|}
\hline
\multicolumn{2}{|l|}{\textit{MAUS will write the following variables to the G4BL configuration file}} \\
\hline
q\_1, q\_2, q\_3, d\_1, d\_s, d\_2 & Field gradient of magnet \\
\hline
particles\_per\_spill & No. of particles to take out of buffer for each spill, if set to zero then all particles are taken from buffer for first spill \\
\hline
run\_number & When retrieving magnet currents and proton absorber thickness from CDB set to the run number of interest \\
\hline
rotation\_angle & Rotation of MAUS co-ordinate system clockwise around the y-axis with respect to G4BL co-ordinate system \\
\hline
translation\_z & The distance between the MAUS centre and the G4BL centre. It assumes the G4BL centre is in front of the MAUS centre \\
\hline
proton\_abserober\_thickness & Thickness of the proton absorber \\
\hline
proton\_number & No. of protons on target in G4BL \\
\hline
proton\_weight & Scales the number of protons generated, with default setting protons are NOT generated \\
\hline
particle\_charge & Refers to the charge of the simulated particles. Can be set to, positive-only, negative-only or all \\
\hline
file\_path & Path to G4BL input \\
\hline
get\_magnet\_currents\_pa\_cdb & If set to True all magnet currents and proton absorber thickness will be retrieved from the CDB and written to the G4BL configuration file for the run number given in this dictionary \\
\hline
random\_seed & Sets the random seed for G4beamline \\
\hline 
\end{tabular}
\label{table}
\end{table}

\paragraph{}

The default configuration variables simulate a 6$\pi$ 200 MeV/c positive beam using a point 1 m downstream of D2 as the interface point. This can be used as input for the MAUS Step IV geometry provided by the Geometry group. To generate MAUS primaries for beams of different momenta or at difference interfaces (i.e. for different MAUS geometries) these variables must be changed accordingly. 

\paragraph{}

The output of this mapper is a json document of MAUS primaries. This is passed directly to MapCppSimulation and so simulatations of the entire MICE beam line from end-to-end can be run. However given the requisite time required to complete such a simulation this is not recommended. Large scale production jobs will be run on the Grid using this mapper to create beam libraries. These are publically available from \url{http://www.ppe.gla.ac.uk/~jnugent/Grid_Files/}. Download the relevant beam file which will contain a list of Grid addresses for files stored on a Grid storage element (SE). You will be required to either run a Grid job to access these files or pull them from the SE to your local area to work on them. For information on working with the Grid see \url{https://www.gridpp.ac.uk/php/support/otherscihelp.php#3}.  

\paragraph{}

The output json document from this mapper is called G4BLoutput.json and is  written in whichever directory \verb+simulate\_mice\_G4BL.py+ was run. To run MAUS using this file as input the script \verb+simulate\_beam.py+ can be used. The path to G4BLoutput.json must be set in the MAUS datacard using the variable \verb+input_json_file_name+. 



















