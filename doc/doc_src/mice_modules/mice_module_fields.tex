\chapter{Field Properties}
Invoke a field using PropertyString FieldType {\textless}fieldtype{\textgreater}. The field will be placed, normally
centred on the MiceModule Position and with the appropriate Rotation. Further options for each field type are specified
below, and relevant datacards are also given. If a fieldtype is specified some other properties must also be specified,
while others may be optional, usually taking their value from defaults specified in the datacards. Only one fieldtype
can be specified per module. However, fields from multiple modules are superimposed, each transformed according to the
MiceModule specification. This enables many different field configurations to be simulated using MAUS.

To use BeamTools fields, datacard FieldMode Full must be set. This is the default.

\begin{center}
\tablefirsthead{\hline
{\bfseries Property} &
{\bfseries Type} &
{\bfseries Description}\\}
\tablehead{\hline
{\bfseries Property} &
{\bfseries Type} &
{\bfseries Description}\\}
\tabletail{}
\tablelasttail{}
\begin{supertabular}{|m{3.665cm}|m{1.278cm}|m{12.052cm}|}
\hline
{\itshape FieldType} &
string &
Set the field type of the MiceModule.\\\hline
\end{supertabular}
\end{center}
\subsection{FieldType CylindricalField}
Sets a constant magnetic field in a cylindrical region symmetric about the z-axis of the module.

\begin{center}
\tablefirsthead{\hline
{\bfseries Property} &
{\bfseries Type} &
{\bfseries Description}\\}
\tablehead{\hline
{\bfseries Property} &
{\bfseries Type} &
{\bfseries Description}\\}
\tabletail{}
\tablelasttail{}
\begin{supertabular}{|m{3.665cm}|m{1.278cm}|m{12.052cm}|}
\hline
{\itshape ConstantField} &
Hep3

Vector &
The magnetic field that will be placed in the region.\\\hline
\multicolumn{2}{m{5.143cm}|}{\hspace*{-\tabcolsep}\begin{tabular}{|m{3.665cm}|m{1.278cm}}

{\itshape Length} &
double\\\hline
{\itshape FieldRadius} &
double\\\hline
\end{tabular}\hspace*{-\tabcolsep}
} &
The physical extent of the region.\\\hhline{~~-}
\end{supertabular}
\end{center}
\subsection{FieldType RectangularField}
Sets a constant magnetic field in a rectangular region.

\begin{center}
\tablefirsthead{\hline
{\bfseries Property} &
{\bfseries Type} &
{\bfseries Description}\\}
\tablehead{\hline
{\bfseries Property} &
{\bfseries Type} &
{\bfseries Description}\\}
\tabletail{}
\tablelasttail{}
\begin{supertabular}{|m{3.665cm}|m{1.278cm}|m{12.052cm}|}
\hline
{\itshape ConstantField} &
Hep3

Vector &
The magnetic field that will be placed in the region.\\\hline
\multicolumn{2}{m{5.143cm}|}{\hspace*{-\tabcolsep}\begin{tabular}{|m{3.665cm}|m{1.278cm}}

{\itshape Length} &
double\\\hline
{\itshape Width} &
double\\\hline
{\itshape Height} &
double\\\hline
\end{tabular}\hspace*{-\tabcolsep}
} &
The physical extent of the region.\\\hhline{~~-}
\end{supertabular}
\end{center}
\subsection{FieldType Solenoid}
MAUS simulates solenoids using a series of current sheets. The field for each solenoid is written to a field map on a
rectangular grid and can then be reused. The field from each current sheet is calculated using the formula for current
sheets detailed in MUCOOL Note 281, \textit{Modeling solenoids using coil, sheet and block conductors}.

\begin{center}
\tablefirsthead{\hline
{\bfseries Property} &
{\bfseries Type} &
{\bfseries Description}\\}
\tablehead{\hline
{\bfseries Property} &
{\bfseries Type} &
{\bfseries Description}\\}
\tabletail{}
\tablelasttail{}
\begin{supertabular}{|m{3.665cm}|m{1.278cm}|m{12.052cm}|}
\hline
{\itshape FileName} &
string &
Read or write solenoid data to the filename. If different modules have the same filename, MAUS assumes they are the
same.\\\hline
FieldMapMode &
string &
If set to Read, MAUS will attempt to read existing data from the FileName. If set to Write, MAUS will generate new
data and write it to the FileName. If set to Analytic, MAUS will calculate fields directly without interpolating. If
set to WriteDynamic acts as in Write except the grid extent and grid spacing at each point is chosen dynamically to
some tolerance defined in the FieldTolerance property. Takes default from datacard SolDataFiles (Write).\\\hline
\multicolumn{2}{m{5.143cm}|}{\hspace*{-\tabcolsep}\begin{tabular}{|m{3.665cm}|m{1.278cm}}

{\itshape Length} &
double\\\hline
{\itshape Thickness} &
double\\\hline
{\itshape InnerRadius} &
double\\\hline
{\itshape CurrentDensity} &
double\\\hline
\end{tabular}\hspace*{-\tabcolsep}
} &
Coil physical parameters. Only used in Write/Analytic mode where they are mandatory.\\\hhline{~~-}
ZExtentFactor &
double &
Field map extends to length + ZExtentFactor*innerRadius in Write mode. Takes default from datacard SolzMapExtendFactor
(10.). Map size is chosen dynamically in WriteDynamic mode.\\\hline
RExtentFactor &
double &
Field map extends to radius RExtentFactor*innerRadius in Write mode. Takes default from datacard SolrMapExtendFactor
(2.018...). Avoid allowing grid nodes to fall on sheets.\\\hline
NumberOfZCoords &
int &
Number of coordinates in \textit{z} in field map grid in Write mode. Takes default from datacard NumberNodesZGrid
(500).\\\hline
NumberOfRCoords &
int &
Number of coordintes in \textit{r} in field map grid in Write mode. Takes default from datacard NumberNodesRGrid
(100).\\\hline
NumberOfSheets &
int &
Number of sheets used to calculate the field. Takes default from datacard DefaultNumberOfSheets (10).\\\hline
{\itshape FieldTolerance } &
double &
Mandatory when FieldMapMode is WriteDynamic. If field map mode is write dynamic, this datacard controls the tolerance on
errors in the field with which the field grid and the grid extent will be chosen. \\\hline
Interpolation

Algorithm &
string &
Choose the interpolation algorithm. Options are BiLinear for a linear interpolation in \textit{r} and \textit{z}, or
LinearCubic \ for a linear interpolation in \textit{r} and a cubic spline in \textit{z}. Default is
LinearCubic.\\\hline
IsAmalgamated &
bool &
Set to 1 to add the coil to CoilAmalgamtion parent field (see below).\\\hline
\end{supertabular}
\end{center}

\subsection{FieldType FieldAmalgamation}
During tracking, MAUS stores a list of fields and for each one MAUS checks to see if tracking is performed through a
particular field map's bounding box. This can be slow if a large number of fields are present. One way to speed this up
is to store contributions from many coils in a single CoilAmalgamation. A CoilAmalgamation searches through child
modules for solenoids with PropertyBool IsAmalgamated set to true. If it finds such a coil, it will add the field
generated by the solenoid to its own field map and disable the coil.

\begin{center}
\tablefirsthead{\hline
{\bfseries Property} &
{\bfseries Type} &
{\bfseries Description}\\}
\tablehead{\hline
{\bfseries Property} &
{\bfseries Type} &
{\bfseries Description}\\}
\tabletail{}
\tablelasttail{}
\begin{supertabular}{|m{3.665cm}|m{1.278cm}|m{12.052cm}|}
\hline
{\itshape Length} &
double &
The Length of the field map generated by the CoilAmalgamation.\\\hline
{\itshape RMax} &
double &
The maximum radius of the field map generated by the CoilAmalgamation.\\\hline
Interpolation

Algorithm &
string &
Choose the interpolation algorithm. Options are BiLinear for a linear interpolation in \textit{r} and \textit{z}, or
LinearCubic \ for a linear interpolation in \textit{r} and a cubic spline in \textit{z}. Default is
LinearCubic.\\\hline
\multicolumn{2}{m{5.143cm}|}{\hspace*{-\tabcolsep}\begin{tabular}{|m{3.665cm}|m{1.278cm}}

{\itshape ZStep} &
double\\\hline
{\itshape RStep} &
double\\\hline
\end{tabular}\hspace*{-\tabcolsep}
} &
Step size of the field map generated by the CoilAmalgamation.\\\hhline{~~-}
\end{supertabular}
\end{center}

\subsection{FieldType DerivativesSolenoid}
This is an alternative field model for solenoids that uses a power law expansion of the on-axis magnetic field and its
derivatives, and an exponential fall-off for the fringe field. The fringe field is defined in the same way as other end
fields, but note that HardEdged end field type is not available for solenoids and will result in an error.

\begin{center}
\tablefirsthead{\hline
{\bfseries Property} &
{\bfseries Type} &
{\bfseries Description}\\}
\tablehead{\hline
{\bfseries Property} &
{\bfseries Type} &
{\bfseries Description}\\}
\tabletail{}
\tablelasttail{}
\begin{supertabular}{|m{3.665cm}|m{1.278cm}|m{12.052cm}|}
\hline
{\itshape PeakField} &
double &
Nominal peak field of the solenoid.\\\hline
{\itshape ZMax} &
double &
Maximum z-half length of the solenoid bounding box in the local coordinate system of the magnet.\\\hline
{\itshape RMax} &
double &
Maximum radius of the solenoid bounding box in the local coordinate system of the magnet.\\\hline
{\itshape MaxEndPole} &
int &
Maximum derivative used in calculating the end field of the solenoid.\\\hline
\end{supertabular}
\end{center}
\subsection{Phasing Models}
MAUS has a number of models for phasing RF cavities.

When CavityMode is Unphased, MAUS attempts to phase the cavity itself. When using CavityMode Unphased MAUS needs to
know when particles enter, cross the middle, and leave cavities. To phase a cavity, MAUS builds a virtual detector in the
centre of the cavity that is used for phasing and then fires a reference particle through the system. Stochastic
processes are always disabled during this process, while mean energy
loss can be disabled using the datacard ReferenceEnergyLossModel. If a reference particle crosses a plane through the
centre of a cavity, it sets the phase of the cavity to the time at which the particle crosses. 

The field of the cavity during phasing is controlled by the property FieldDuringPhasing. There are four modes:

\liststyleLxiv
\begin{itemize}
\item \textit{None}: Cavity fields are disabled during phasing
\item \textit{Electrostatic}: An electrostatic field with no positional dependence given by
PeakEField*sin(ReferenceParticlePhase) is enabled during phasing.
\item \textit{TimeVarying}: A standard time varying field is applied during phasing, initially with arbitrary phase
relative to the reference particle. MAUS uses a Newton-Raphson method to find the appropriate reference phase
iteratively, with tolerance set by the datacard PhaseTolerance.
\item \textit{EnergyGainOptimised}: A standard time varying field is applied during phasing, initially with arbitrary
phase and peak field relative to the reference particle. MAUS uses a 2D Newton-Raphson method to find the appropriate
reference phase and peak field iteratively, so that the reference particle crosses the cavity centre with phase given
by property ReferenceParticlePhase and gains energy over the whole cavity given by property EnergyGain with tolerances
set by the datacards PhaseTolerance and RFDeltaEnergyTolerance.
\end{itemize}
\subsection{Tracking Stability Around RF Cavities}
Usually RF cavities have little or no fringe field, and this can lead to some instability in the tracking algorithms.
When MAUS makes a step into an RF cavity volume, the tracking algorithms tend to smooth out a field in a non-physical
way. This can be prevented by either (i) making the step size rather small in the RF cavity or (ii) forcing MAUS to
stop tracking by adding a physical volume at the entrance of the RF cavity (a window, typically made of vacuum). Doing
this should improve stability of tracking.

\subsection{FieldType PillBox}
Sets a PillBox field in a particular region. MAUS represents pillboxes using a sinusoidally varying TM010 pill box
field, with non-zero field vector elements given by

\begin{equation*}
\begin{gathered}B_{\phi }=J_1(k_rr)\cos (\omega t)\\E_z=J_0(k_rr)\cos (\omega t)\end{gathered}
\end{equation*}
Here J\textsubscript{n} are Bessel functions and k\textsubscript{r} is a constant. See, for example, SY Lee VI.1. All
other fields are 0.

\begin{center}
\tablefirsthead{\hline
{\bfseries Property} &
{\bfseries Type} &
{\bfseries Description}\\}
\tablehead{\hline
{\bfseries Property} &
{\bfseries Type} &
{\bfseries Description}\\}
\tabletail{}
\tablelasttail{}
\begin{supertabular}{|m{3.665cm}|m{1.278cm}|m{12.052cm}|}
\hline
{\itshape Length} &
double &
Length of the region in which the field is present.\\\hline
{\itshape CavityMode} &
string &
Phasing mode of the cavity - options are Phased, Unphased and Electrostatic.\\\hline
{\itshape FieldDuringPhasing} &
string &
Controls the field during cavity phasing -- options are None, Electrostatic, TimeVarying and
EnergyGainOptimised.\\\hline
{\itshape EnergyGain} &
double &
WhenFieldDuringPhasing is set to EnergyGainOptimised, controls the peak electric field.\\\hline
{\itshape Frequency} &
double &
The cavity frequency.\\\hline
{\itshape PeakEField} &
double &
The peak field of the cavity. Not used when the FieldDuringPhasing is EnergyGainOptimised.\\\hline
{\itshape TimeDelay} &
double &
In Phased mode the time delay (absolute time) of the cavity.\\\hline
PhasingVolume &
string &
Set to SpecialVirtual to make the central volume a special virtual.\\\hline
\multicolumn{2}{m{5.143cm}|}{\hspace*{-\tabcolsep}\begin{tabular}{|m{3.665cm}|m{1.278cm}}

ReferenceParticle

Energy &
double\\\hline
ReferenceParticle

Charge &
double\\\hline
\end{tabular}\hspace*{-\tabcolsep}
} &
In Electrostatic mode, MAUS calculates the peak field and the field the reference particle sees using a combination of
the reference particle energy, charge and phase. Take defaults from datacards NominalKineticEnergy and MuonCharge
\\\hhline{~~-}
ReferenceParticle

Phase &
double &
MAUS tries to phase the field so that the reference particle crosses the cavity at ReferenceParticlePhase (units are
angular). 0\textsuperscript{o} corresponds to no energy gain, 90\textsuperscript{o} corresponds to operation on-crest.
Default from datacard rfAcclerationPhase.\\\hline
\end{supertabular}
\end{center}
\subsection{FieldType RFFieldMap}
Sets a cavity with an RF field map in a particular region. RFFieldMap uses the same phasing algorithm as described
above.

\begin{center}
\tablefirsthead{\hline
{\bfseries Property} &
{\bfseries Type} &
{\bfseries Description}\\}
\tablehead{\hline
{\bfseries Property} &
{\bfseries Type} &
{\bfseries Description}\\}
\tabletail{}
\tablelasttail{}
\begin{supertabular}{|m{3.665cm}|m{1.278cm}|m{12.052cm}|}
\hline
{\itshape Length} &
double &
Length of the region in which the field is present.\\\hline
{\itshape CavityMode} &
string &
Phasing mode of the cavity - options are Phased and Unphased. RFFieldMaps cannot operated in Electrostatic mode.\\\hline
{\itshape FieldDuringPhasing} &
string &
Controls the field during cavity phasing -- options are None, Electrostatic, TimeVarying and
EnergyGainOptimised.\\\hline
{\itshape EnergyGain} &
double &
WhenFieldDuringPhasing is set to EnergyGainOptimised, controls the peak electric field.\\\hline
{\itshape Frequency} &
double &
The cavity frequency.\\\hline
{\itshape PeakEField} &
double &
The peak field of the cavity. Not used when the FieldDuringPhasing is EnergyGainOptimised.\\\hline
{\itshape TimeDelay} &
double &
In Phased mode the time delay (absolute time) of the cavity.\\\hline
PhasingVolume &
string &
Set to SpecialVirtual to make the central volume a special virtual.\\\hline
\multicolumn{2}{m{5.143cm}|}{\hspace*{-\tabcolsep}\begin{tabular}{|m{3.665cm}|m{1.278cm}}

ReferenceParticle

Energy &
double\\\hline
ReferenceParticle

Charge &
double\\\hline
\end{tabular}\hspace*{-\tabcolsep}
} &
In Electrostatic mode, MAUS calculates the peak. field and the field the reference particle sees using a combination
of the reference particle energy, charge and phase. Take defaults from datacards NominalKineticEnergy and MuonCharge
\\\hhline{~~-}
ReferenceParticle

Phase &
double &
MAUS tries to phase the field so that the reference particle crosses the cavity at ReferenceParticlePhase (units are
angular). 0\textsuperscript{o} corresponds to no energy gain, 90\textsuperscript{o} corresponds to operation on-crest.
Default from datacard rfAcclerationPhase.\\\hline
{\itshape FileName} &
string &
The file name of the field map file.\\\hline
{\itshape FileType} &
string &
The file type of the field map. Only supported option is SuperFishSF7.\\\hline
\end{supertabular}
\end{center}

\subsection{FieldType Multipole}
Creates a multipole of arbitrary order. Fields are generated using either a hard edged model, with no fringe fields at
all; or an Enge model similar to ZGoubi and COSY. In the former case fields are calculated using a simple polynomial
expansion. In the latter case fields are calculated using the polynomial expansion with an additional exponential drop
off. Fields can be superimposed onto a bent coordinate system to generate a sector multipole with arbitrary fixed
radius of curvature.

Unlike most other field models in MAUS, the zero position corresponds to the center of the entrance of the multipole;
and the multipole extends in the +z direction.

The method to define end fields is described in the section EndFieldTypes below

\begin{center}
\tablefirsthead{\hline
{\bfseries Property} &
{\bfseries Type} &
{\bfseries Description}\\}
\tablehead{\hline
{\bfseries Property} &
{\bfseries Type} &
{\bfseries Description}\\}
\tabletail{}
\tablelasttail{}
\begin{supertabular}{|m{3.691cm}|m{1.252cm}|m{12.052cm}|}
\hline
{\itshape Pole} &
int &
The reference pole of the magnet. 1=dipole, 2=quadrupole, 3=sextupole etc.\\\hline
{\itshape FieldStrength} &
double &
Scale the field strength in the good field region. For dipoles, this sets the dipole field; for quadrupoles this sets
the field gradient. Note that for some end field settings there can be no good field region (e.g. if the end length is
{\textgreater}\~{} centre length).\\\hline
{\itshape Height} &
double &
Height of the field region.\\\hline
{\itshape Width} &
double &
Width or delta radius of the field region.\\\hline
{\itshape Length} &
double &
Length of the field along the bent trajectory.\\\hline
EndFieldType &
string &
Set to HardEdged to disable fringe fields. Set to Enge or Tanh to use those models, as described elsewhere. Default is
HardEdged.\\\hline
CurvatureModel &
string &
Choose the model for curvature. Straight forces no curvature. Constant gives a constant radius of curvature;
StraightEnds gives a constant radius of curvature along the length of the multipole with straight end fields beyond
this length. MomentumBased gives radius of curvature determined by a momentum and a total bending angle.\\\hline
ReferenceCurvature &
double &
Radius of curvature of the magnet in Constant or StraightEnds mode. Set to 0 for a straight magnet. Default is
0.\\\hline
ReferenceMomentum &
double &
Reference momentum used to calculate the radius of curvature of a dipole in MomentumBased mode. Default is 0.\\\hline
{\itshape BendingAngle} &
double &
The angle used to calculate the radius of curvature of a dipole in MomentumBased mode. Note that this is mandatory in
MomentumBased mode.\\\hline
\end{supertabular}
\end{center}

\subsection{FieldType CombinedFunction}
This creates superimposed dipole, quadrupole and sextupole fields with a common radius of curvature. The field is
intended to mimic the first few terms in a multipole expansion of a field like

\begin{equation*}
B(y=0)=B_0\left(\frac r{r_0}\right)^k
\end{equation*}
The field index is a user defined parameter, while the dipole field and radius of curvature can either be defined
directly by the user or defined in terms of a reference momentum and total bending angle. Fields are calculated as in
the multipole field type defined above.

\begin{center}
\tablefirsthead{\hline
{\bfseries Property} &
{\bfseries Type} &
{\bfseries Description}\\}
\tablehead{\hline
{\bfseries Property} &
{\bfseries Type} &
{\bfseries Description}\\}
\tabletail{}
\tablelasttail{}
\begin{supertabular}{|m{3.689cm}|m{1.252cm}|m{12.054cm}|}
\hline
{\itshape Pole} &
int &
The reference pole of the magnet. 1=dipole, 2=quadrupole, 3=sextupole etc.\\\hline
{\itshape BendingField} &
double &
The nominal dipole field \textit{B}\textit{\textsubscript{0}}. Note that this is mandatory in all cases except where
CurvatureModel is MomentumBased, when the BendingAngle and ReferenceMomentum is used to calculate the \ dipole field
instead.\\\hline
{\itshape FieldIndex} &
double &
The field index \textit{k}.\\\hline
{\itshape Height} &
double &
Height of the field region.\\\hline
{\itshape Width} &
double &
Width or delta radius of the field region.\\\hline
{\itshape Length} &
double &
Length of the field along the bent trajectory.\\\hline
EndFieldType &
string &
Set to HardEdged to disable fringe fields. Set to Enge or Tanh to use those models, as described elsewhere. Default is
HardEdged.\\\hline
CurvatureModel &
string &
Choose the model for curvature. Straight forces no curvature. Constant gives a constant radius of curvature;
StraightEnds gives a constant radius of curvature along the length of the multipole with straight end fields beyond
this length. MomentumBased gives radius of curvature determined by a momentum and a total bending angle.\\\hline
ReferenceCurvature &
double &
Radius of curvature of the magnet in Constant or StraightEnds mode. Set to 0 for a straight magnet. Default is
0.\\\hline
ReferenceMomentum &
double &
Reference momentum used to calculate the radius of curvature of a dipole in MomentumBased mode. Default is 0.\\\hline
{\itshape BendingAngle} &
double &
The angle used to calculate the radius of curvature of a dipole in MomentumBased mode. Note that this is mandatory in
MomentumBased mode.\\\hline
\end{supertabular}
\end{center}

\subsection{EndFieldTypes}
In the absence of current sources, the magnetic field can be calculated from a scalar potential using the standard
relation

\begin{equation*}
\vec B=\nabla V_n
\end{equation*}
The scalar magnetic potential of the n\textsuperscript{th}{}-order multipole field is given by

\begin{equation*}
V_n=\sum _{q=0}^{q_m}\sum _{m=0}^nn!^2\frac{G^{(2q)}(s)(r^2+y^2)^q\sin (\frac{m\pi } 2)r^{n-m}y^m}{4^qq!(n+q)!m!(n-m)!}
\end{equation*}
where \textit{G(s)} is either the double Enge function,

\begin{equation*}
G(s)=E[(x-x_0)/\lambda ]+E[(-x-x_0)/\lambda ]-1
\end{equation*}
\begin{equation*}
E(s)=\frac{B_0}{R_0^n}\frac 1{1+\exp (C_1+C_2s+C_3s^2+...)}
\end{equation*}
or G(s) is the double tanh function,

\begin{equation*}
G(s)=\tanh [(x+x_0)/\lambda ]/2+\tanh [(x-x_0)/\lambda ]/2
\end{equation*}
and \textit{(r, y, s)} is the position vector in the rotating coordinate system. Note that here s is the distance from
the nominal end of the field map.

\begin{center}
\tablefirsthead{\hline
{\bfseries Property} &
{\bfseries Type} &
{\bfseries Description}\\}
\tablehead{\hline
{\bfseries Property} &
{\bfseries Type} &
{\bfseries Description}\\}
\tabletail{}
\tablelasttail{}
\begin{supertabular}{|m{3.689cm}|m{1.252cm}|m{12.054cm}|}
\hline
EndFieldType &
string &
Set to HardEdged to disable fringe fields. Set to Enge or Tanh to use those models, as described elsewhere. Default is
HardEdged.\\\hline
\multicolumn{3}{|m{17.394999cm}|}{{\itshape The following properties are used for EndFieldType Tanh}}\\\hline
EndLength &
double &
Set the l parameter that defines the rapidity of the field fall off.\\\hline
CentreLength &
double &
Set the \textit{x}\textit{\textsubscript{0}} parameter that defines the length of the flat field region.\\\hline
MaxEndPole &
int &
Set the maximum pole that will be calculated -- should be larger than the multipole pole.\\\hline
\multicolumn{3}{|m{17.394999cm}|}{{\itshape The following properties are used for EndFieldType Enge}}\\\hline
EndLength &
double &
Set the l parameter that defines the rapidity of the field fall off.\\\hline
CentreLength &
double &
Set the \textit{x}\textit{\textsubscript{0}} parameter that defines the length of the flat field region.\\\hline
MaxEndPole &
int &
Set the maximum pole that will be calculated -- should be larger than the multipole pole.\\\hline
\multicolumn{2}{m{5.1410003cm}|}{\hspace*{-\tabcolsep}\begin{tabular}{|m{3.689cm}|m{1.252cm}}

Enge1 &
double\\\hline
Enge2 &
double\\\hline
... &
double\\\hline
EngeN &
double\\\hline
\end{tabular}\hspace*{-\tabcolsep}
} &
Set the parameters C\textsubscript{i} as defined in the Enge function above.\\\hhline{~~-}
\end{supertabular}
\end{center}
\subsection{FieldType MagneticFieldMap}
Reads or writes a magnetic field map in a particular region. Two sorts of field maps are supported; either a 2d field
map, in which cylindrical symmetry is assumed, or a 3d field map. 

For 2d field maps, MAUS reads or writes a file that contains information about the radial and longitudinal field
components. This is intended for solenoidal field maps where only radial and longitudinal field components are present.
Note that in write mode, MAUS assumes cylindrical symmetry of the fields. In this case, MAUS writes the \textit{x}
and \textit{z} components of the magnetic field at points on a grid in \textit{x} and \textit{z}. Fields with an
electric component are excluded from this summation.

For 3d field maps, MAUS reads a file that contains the position and field in cartesian coordinates and performs a
linear interpolation. This requires quite large field map files; the file size can be slightly reduced by using certain
symmetries, as described below. It is currently not possible to write 3d field maps.

\begin{center}
\tablefirsthead{\hline
{\bfseries Property} &
{\bfseries Type} &
{\bfseries Description}\\}
\tablehead{\hline
{\bfseries Property} &
{\bfseries Type} &
{\bfseries Description}\\}
\tabletail{}
\tablelasttail{}
\begin{supertabular}{|m{3.665cm}|m{1.278cm}|m{12.052cm}|}
\hline
{\itshape FieldMapMode} &
string &
Set to Read to read a field map; and Write to write a field map.\\\hline
{\itshape FileName} &
string &
The file name that is used for reading or writing.\\\hline
FileType &
string &
The file format. Supported options in Read mode are MAUStext, MAUSbinary, g4beamline, icool, g4bl3dGrid. Only
MAUStext is supported in Write mode. Default is MAUStext.\\\hline
Symmetry &
string &
Symmetry option for g4bl3dGrid file type. Options are None, Dipole or Quadrupole. None uses the field map as is, while
Dipole and Quadrupole reflect the octant between the positive \textit{x}, \textit{y} and \textit{z} axes to give an
appropriate field for a dipole or quadrupole.\\\hline
\multicolumn{2}{m{5.143cm}|}{\hspace*{-\tabcolsep}\begin{tabular}{|m{3.665cm}|m{1.278cm}}

{\itshape ZStep} &
double\\\hline
{\itshape RStep} &
double\\\hline
\end{tabular}\hspace*{-\tabcolsep}
} &
Step size in \textit{z} and \textit{r}. Mandatory in Write mode but not used in Read mode (where step size comes from
the map file).\\\hhline{~~-}
\multicolumn{2}{m{5.143cm}|}{\hspace*{-\tabcolsep}\begin{tabular}{|m{3.665cm}|m{1.278cm}}

{\itshape ZMin} &
double\\\hline
{\itshape ZMax} &
double\\\hline
{\itshape RMin} &
double\\\hline
{\itshape RMax} &
double\\\hline
\end{tabular}\hspace*{-\tabcolsep}
} &
Upper and lower bounds in \textit{z} and \textit{r}. Mandatory in Write mode but not used in Read mode (where boundaries
come from the map file).\\\hhline{~~-}
\end{supertabular}
\end{center}
Some file formats are described below. I am working towards making the file format more generic and hence possibly
easier to use, but backwards compatibility will hopefully be maintained. 

\subsubsection{MAUStext Field Map Format}
The native field map format used by MAUS in text mode is described below.

{\ttfamily
\# GridType = Uniform N = \textbf{number\_rows}}

{\ttfamily
\# Z1 = \textbf{z\_start} Z2 = \textbf{z\_end} dZ = \textbf{z\_step}}

{\ttfamily
\# R1 = \textbf{r\_start} R2 = \textbf{r\_end} dR = \textbf{r\_step}}

{\ttfamily\bfseries
Bz\_Values\ \ Br\_Values}

{\ttfamily\bfseries
...\ \ \ \ ...}

{\ttfamily\bfseries
{\textless}Repeat as necessary{\textgreater}}

In this mode, field maps are represented by field values on a regular 2d grid that is assumed to have solenoidal
symmetry, i.e. cylindrical symmetry with no tangential component.

\begin{center}
\tablefirsthead{\hline
{\bfseries Name} &
{\bfseries Type} &
{\bfseries Description}\\}
\tablehead{\hline
{\bfseries Name} &
{\bfseries Type} &
{\bfseries Description}\\}
\tabletail{}
\tablelasttail{}
\begin{supertabular}{|m{3.665cm}|m{2.305cm}|m{11.025001cm}|}
\hline
{\ttfamily\bfseries number\_rows} &
double &
Number of rows in the field map file.\\\hline
{\ttfamily\bfseries z\_start} &
double &
Position of the grid start along the \textit{z} axis.\\\hline
{\ttfamily\bfseries z\_end} &
double &
Position of the grid end along the \textit{z} axis.\\\hline
{\ttfamily\bfseries z\_step} &
double &
Step size in \textit{z}.\\\hline
{\ttfamily\bfseries r\_start} &
double &
Position of the grid start along the \textit{r} axis.\\\hline
{\ttfamily\bfseries r\_end} &
double &
Position of the grid end along the \textit{r} axis.\\\hline
{\ttfamily\bfseries r\_step} &
double &
Step size in \textit{r}.\\\hline
{\ttfamily\bfseries Bz\_Values} &
double &
\textit{Bz} field value.\\\hline
{\ttfamily\bfseries Br\_Values} &
double &
\textit{Br} field value.\\\hline
\end{supertabular}
\end{center}
\subsubsection{g4bl3dGrid Field Map Format}
The file format for 3d field maps is a slightly massaged version of a file format used by another code, g4beamline. In
this mode, field maps are represented by field values on a regular cartesian 3d grid.

{\ttfamily\bfseries
number\_x\_points number\_y\_points number\_z\_points global\_scale}

{\ttfamily
1 X [\textbf{x\_scale}]}

{\ttfamily
2 Y [\textbf{y\_scale}]}

{\ttfamily
3 Z [\textbf{z\_scale}]}

{\ttfamily
4 BX [\textbf{bx\_scale}]}

{\ttfamily
5 BY [\textbf{by\_scale}]}

{\ttfamily
6 BZ [\textbf{bz\_scale}]}

{\ttfamily
0}

{\ttfamily\bfseries
X\_Values \ \ Y\_Values\ \ Z\_Values\ \ Bx\_values\ \ By\_values\ \ Bz\_values}

{\ttfamily\bfseries
...\ \ \ \ ...\ \ \ \ ...\ \ \ \ ...\ \ \ \ ...\ \ \ \ ...}

{\ttfamily\bfseries
{\textless}Repeat as necessary{\textgreater}}

where text in bold indicates a value described in the following table

\begin{center}
\tablefirsthead{\hline
{\bfseries Name} &
{\bfseries Type} &
{\bfseries Description}\\}
\tablehead{\hline
{\bfseries Name} &
{\bfseries Type} &
{\bfseries Description}\\}
\tabletail{}
\tablelasttail{}
\begin{supertabular}{|m{3.665cm}|m{2.305cm}|m{11.025001cm}|}
\hline
{\ttfamily\bfseries number\_x\_points} &
double &
Number of points along x axis.\\\hline
{\ttfamily\bfseries number\_y\_points} &
double &
Number of points along y axis.\\\hline
{\ttfamily\bfseries number\_z\_points} &
double &
Number of points along z axis.\\\hline
{\ttfamily\bfseries global\_scale} &
double &
Global scale factor applied to all x, y, z and Bx, By, Bz values.\\\hline
{\ttfamily\bfseries x\_scale} &
double &
Scale factor applied to all x values.\\\hline
{\ttfamily\bfseries y\_scale} &
double &
Scale factor applied to all y values.\\\hline
{\ttfamily\bfseries z\_scale} &
double &
Scale factor applied to all z values.\\\hline
{\ttfamily\bfseries bx\_scale} &
double &
Scale factor applied to all Bx values.\\\hline
{\ttfamily\bfseries by\_scale} &
double &
Scale factor applied to all By values.\\\hline
{\ttfamily\bfseries bz\_scale} &
double &
Scale factor applied to all Bz values.\\\hline
{\ttfamily\bfseries X\_Values} &
double &
List (column) of each x value.\\\hline
{\ttfamily\bfseries Y\_Values} &
double &
List (column) of each y value.\\\hline
{\ttfamily\bfseries Z\_Values} &
double &
List (column) of each z value.\\\hline
{\ttfamily\bfseries Bx\_Values} &
double &
List (column) of each Bx value.\\\hline
{\ttfamily\bfseries By\_Values} &
double &
List (column) of each By value.\\\hline
{\ttfamily\bfseries Bz\_Values} &
double &
List (column) of each Bz value.\\\hline
\end{supertabular}
\end{center}
