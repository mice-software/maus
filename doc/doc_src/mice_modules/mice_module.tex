% This file was converted to LaTeX by Writer2LaTeX ver. 1.2
% see http://writer2latex.sourceforge.net for more info
\documentclass[letterpaper]{article}
\usepackage[ascii]{inputenc}
\usepackage[T1]{fontenc}
\usepackage[english]{babel}
\usepackage{amsmath}
\usepackage{amssymb,amsfonts,textcomp}
\usepackage{color}
\usepackage{array}
\usepackage{supertabular}
\usepackage{hhline}
\usepackage{hyperref}
\hypersetup{pdftex, colorlinks=true, linkcolor=blue, citecolor=blue, filecolor=blue, urlcolor=blue, pdftitle=, pdfauthor=, pdfsubject=, pdfkeywords=}
\usepackage[pdftex]{graphicx}
\newcommand\textsubscript[1]{\ensuremath{{}_{\text{#1}}}}
% Outline numbering
\setcounter{secnumdepth}{0}
\makeatletter
\newcommand\arraybslash{\let\\\@arraycr}
\makeatother
% List styles
\newcommand\liststyleLi{%
\renewcommand\labelitemi{${\bullet}$}
\renewcommand\labelitemii{${\circ}$}
\renewcommand\labelitemiii{${\blacksquare}$}
\renewcommand\labelitemiv{${\bullet}$}
}
\newcommand\liststyleLii{%
\renewcommand\labelitemi{${\bullet}$}
\renewcommand\labelitemii{${\circ}$}
\renewcommand\labelitemiii{${\blacksquare}$}
\renewcommand\labelitemiv{${\bullet}$}
}
\newcommand\liststyleLiii{%
\renewcommand\labelitemi{${\bullet}$}
\renewcommand\labelitemii{${\circ}$}
\renewcommand\labelitemiii{${\blacksquare}$}
\renewcommand\labelitemiv{${\bullet}$}
}
\newcommand\liststyleLiv{%
\renewcommand\labelitemi{${\bullet}$}
\renewcommand\labelitemii{${\circ}$}
\renewcommand\labelitemiii{${\blacksquare}$}
\renewcommand\labelitemiv{${\bullet}$}
}
\newcommand\liststyleLv{%
\renewcommand\labelitemi{${\bullet}$}
\renewcommand\labelitemii{${\circ}$}
\renewcommand\labelitemiii{${\blacksquare}$}
\renewcommand\labelitemiv{${\bullet}$}
}
\newcommand\liststyleLvi{%
\renewcommand\labelitemi{${\bullet}$}
\renewcommand\labelitemii{${\circ}$}
\renewcommand\labelitemiii{${\blacksquare}$}
\renewcommand\labelitemiv{${\bullet}$}
}
\newcommand\liststyleLvii{%
\renewcommand\labelitemi{${\bullet}$}
\renewcommand\labelitemii{${\circ}$}
\renewcommand\labelitemiii{${\blacksquare}$}
\renewcommand\labelitemiv{${\bullet}$}
}
\newcommand\liststyleLviii{%
\renewcommand\labelitemi{{\textbullet}}
\renewcommand\labelitemii{${\circ}$}
\renewcommand\labelitemiii{${\blacksquare}$}
\renewcommand\labelitemiv{{\textbullet}}
}
\newcommand\liststyleLix{%
\renewcommand\labelitemi{{\textbullet}}
\renewcommand\labelitemii{${\circ}$}
\renewcommand\labelitemiii{${\blacksquare}$}
\renewcommand\labelitemiv{{\textbullet}}
}
\newcommand\liststyleLx{%
\renewcommand\labelitemi{{\textbullet}}
\renewcommand\labelitemii{${\circ}$}
\renewcommand\labelitemiii{${\blacksquare}$}
\renewcommand\labelitemiv{{\textbullet}}
}
\newcommand\liststyleLxi{%
\renewcommand\theenumi{\arabic{enumi}}
\renewcommand\theenumii{\arabic{enumii}}
\renewcommand\theenumiii{\arabic{enumiii}}
\renewcommand\theenumiv{\arabic{enumiv}}
\renewcommand\labelenumi{\theenumi.}
\renewcommand\labelenumii{\theenumii.}
\renewcommand\labelenumiii{\theenumiii.}
\renewcommand\labelenumiv{\theenumiv.}
}
\newcommand\liststyleLxii{%
\renewcommand\labelitemi{${\bullet}$}
\renewcommand\labelitemii{${\circ}$}
\renewcommand\labelitemiii{${\blacksquare}$}
\renewcommand\labelitemiv{${\bullet}$}
}
\newcommand\liststyleLxiii{%
\renewcommand\labelitemi{${\bullet}$}
\renewcommand\labelitemii{${\circ}$}
\renewcommand\labelitemiii{${\blacksquare}$}
\renewcommand\labelitemiv{${\bullet}$}
}
\newcommand\liststyleLxiv{%
\renewcommand\labelitemi{${\bullet}$}
\renewcommand\labelitemii{${\circ}$}
\renewcommand\labelitemiii{${\blacksquare}$}
\renewcommand\labelitemiv{${\bullet}$}
}
% Page layout (geometry)
\setlength\voffset{-1in}
\setlength\hoffset{-1in}
\setlength\topmargin{2cm}
\setlength\oddsidemargin{2cm}
\setlength\textheight{23.017668cm}
\setlength\textwidth{17.59cm}
\setlength\footskip{26.144882pt}
\setlength\headheight{0cm}
\setlength\headsep{0cm}
% Footnote rule
\setlength{\skip\footins}{0.119cm}
\renewcommand\footnoterule{\vspace*{-0.018cm}\setlength\leftskip{0pt}\setlength\rightskip{0pt plus 1fil}\noindent\textcolor{black}{\rule{0.25\columnwidth}{0.018cm}}\vspace*{0.101cm}}
% Pages styles
\makeatletter
\newcommand\ps@Standard{
  \renewcommand\@oddhead{}
  \renewcommand\@evenhead{}
  \renewcommand\@oddfoot{\thepage{}}
  \renewcommand\@evenfoot{\@oddfoot}
  \renewcommand\thepage{\arabic{page}}
}
\makeatother
\pagestyle{Standard}
\setlength\tabcolsep{1mm}
\renewcommand\arraystretch{1.3}
\title{}
\author{}
\date{2012-03-12}
\begin{document}
\section{Mice Module Documentation}
This document documents the main functionality of the Mice Modules that control geometry and fields of G4MICE. The
document is divided into three parts. Part 1 gives an overview of the basic usage of Mice Modules. Part 2 gives details
on how to implement different geometry models in G4MICE. Part 3 gives details on how to implement different
electromagnetic field models in G4MICE.

\subsection{How to Use this document}
This document is best used in conjunction with the various examples than can be found in any installation of G4MICE. If
you are familiar with Mice Modules already, the first part can be omitted and the latter two parts used as reference
materials.

Any problems or omissions should be emailed to the document author, chris.rogers@stfc.ac.uk. I really like to get
feedback on what is annoying in the documentation, what doesn't make sense and what is just wrong -- so I encourage you
to hassle me!

\clearpage
\bigskip

\setcounter{tocdepth}{10}
\renewcommand\contentsname{Table of Contents}
\tableofcontents

\bigskip

\clearpage\section{Overview}
Mice Modules are the objects that control the geometry and fields that are simulated in G4MICE. They are used in
conjunction with a datacard file, which provides global run control parameters. Mice Modules are created by reading in
a series of text files when G4MICE applications are run. All Mice Module files must be in a subdirectory of the
directory defined by the environment variable \textit{\$\{MICEFILES\}}.

This geometry information is used primarily by the Simulation application for tracking of particles through magnetic
fields. A few commands are specific to detector Reconstruction and accelerator beam Optics applications.

The Mice Modules are created in a tree structure. Each module is a parent of any number of child modules. Typically the
parent module will describe a physical volume, and child modules will describe physical volumes that sit inside the
parent module. Modules cannot be used to describe volumes that do not sit at least partially inside the volume if the
parent module.

Each Mice Module file consists of a series of lines of text. Firstly the Module name is defined. This is followed by an
opening curly bracket, then the description of the module and the placement of any child modules, and finally a closing
curly bracket. Commands, curly brackets etc must be separated by an end of line character.

Comments are indicated using either two slashes or an exclamation mark. Characters placed after a comment on a line are
ignored.

G4MICE operates in a right handed coordinate system \textit{(x,y,z)}. In the absence of any rotation, lengths are
considered to be extent along the \textit{z}{}-axis, widths to be extent along the \textit{x}{}-axis and heights to be
extent along the \textit{y}{}-axis. Rotations (q\textsubscript{x}, q\textsubscript{y}, q\textsubscript{z}) are defined
as a rotation about the z-axis through q\textsubscript{z}, followed by a rotation about the y-axis through
q\textsubscript{y}, followed by a rotation about the x-axis through q\textsubscript{x}.

\subsection{Configuration File}
The Configuration file places the top level objects in MICE. The location of the file is controlled by the datacard
\textit{MiceModel}. G4MICE looks for the configuration file in the first instance in the directory

\ \ \$\textit{\{MICEFILES\}/Models/Configuration/{\textless}MiceModel{\textgreater}}

where \textit{\$\{MICEFILES\}} is a user-defined environment variable. If G4MICE fails to find the file it searches the
local directory.

The world volume is defined in the Configuration file and any children of the world volume are referenced by the
Configuration file. The Configuration file looks like:

{\ttfamily
\ \ Configuration \textit{{\textless}Configuration Name{\textgreater}}}

{\ttfamily
\ \ \{}

{\ttfamily\itshape
\ \ \ \ \textup{Dimensions\ \ }{\textless}x{\textgreater} {\textless}y{\textgreater} {\textless}z{\textgreater}
{\textless}Units{\textgreater}}

{\ttfamily\itshape
\ \ \ \ {\textless}Properties{\textgreater}}

{\ttfamily\itshape
\ \ \ \ {\textless}Child Modules{\textgreater}}

{\ttfamily
\ \ \}}

\textit{{\textless}Configuration Name{\textgreater}} is the name of the configuration. Typically the Configuration file
name is given by \textit{{\textless}Configuration Name{\textgreater}}.dat. The world volume is always a rectangular box
centred on (0,0,0) with x, y, and z extent set by the Dimensions. Properties and Child Modules are described below and
added as in any Module.

\subsubsection{Substitutions}
It is possible to make keyword substitutions that substitutes all instances of {\textless}name{\textgreater} with
{\textless}value{\textgreater} in all Modules. The syntax is

{\ttfamily
\ \ Substitution {\textless}name{\textgreater} {\textless}value{\textgreater}}

{\textless}name{\textgreater} must start with a single \$ sign. Substitutions must be defined in the Configuration file.
Note this is a direct text substitution in the MiceModules before the modules are parsed properly. So for example,

{\ttfamily
\ \ Substitution \$Sub SomeText}

{\ttfamily
\ \ PropertyString FieldType \$Sub}

{\ttfamily
\ \ PropertyDouble \$SubValue 10}

would be parsed as G4MICE like

{\ttfamily
\ \ PropertyString FieldType \ \ \ \ SomeText}

{\ttfamily
\ \ PropertyDouble SomeTextValue 10}

\subsubsection{Expressions}
The use of equations in properties of type double and Hep3Vector is also allowed in place of a single value. So, for
example, 

\ \ \texttt{PropertyDouble FieldStrength 0.5*2 T}

would result in a FieldStrength property of 1 Tesla.\texttt{ }

\subsubsection{Expression Substitutions}
Some additional variables can be defined in specific cases by G4MICE itself for substitution into experssions, in which
case they will start with @ symbol. For these variable substitutions, it is only possible to make the substitution into
expressions. So for example,

{\ttfamily
\ \ PropertyDouble ScaleFactor 2*@RepeatNumber}

Would substitute @RepeatNumber \ into the expression. @RepeatNumber \ is defined by G4MICE when repeating modules are
used (see RepeatModule2, below). Note the following code is not valid

{\ttfamily
\ \ PropertyString FileName File@RepeatNumber //NOT VALID}

This is because Expression Substitutions can only be used in an expression (i.e. an equation).

\subsection[Module Files]{Module Files}
Children of the top level Mice Module are defined by Modules. G4MICE will attempt to find a module in an external file.
The location of the file is controlled by the parent Module. Initially G4MICE looks in the directory

\ \ \$\textit{\{MICEFILES\}/Models/Modules/{\textless}Module{\textgreater}}

If the Mice Module cannot be found, G4MICE searches the local directory. If the module file still cannot be found,
G4MICE will issue a warning and proceed.

The Module description is similar in structure to the Configuration file:

{\ttfamily
\ \ Module \textit{{\textless}Module Name{\textgreater}}}

{\ttfamily
\ \ \{}

{\ttfamily
\ \ \ \ Volume\ \ \textit{{\textless}Volume Type{\textgreater}}}

{\ttfamily\itshape
\ \ \ \ \textup{Dimensions\ \ }{\textless}Dimension1{\textgreater} {\textless}Dimension2{\textgreater}
{\textless}Dimension3{\textgreater} {\textless}Units{\textgreater}}

{\ttfamily\itshape
\ \ \ \ {\textless}Properties{\textgreater}}

{\ttfamily\itshape
\ \ \ \ {\textless}Child Modules{\textgreater}}

{\ttfamily
\ \ \}}

\textit{{\textless}Module Name{\textgreater}} is the name of the module. Typically the Module file name is given by
\textit{{\textless}Module Name{\textgreater}}.dat.

The definition of Volume, Dimensions, Properties and Child Modules are described below.

\subsection{Volume and Dimensions}
The volume described by the MiceModule can be one of several types. Replace \textit{{\textless}Volume
Type{\textgreater}} with the appropriate volume below. Cylinder, Box and Tube define cylindrical and cuboidal volumes.
Polycone defines an arbitrary volume of rotation and is described in detail below. Wedge describes a wedge with a
triangular projection in the y-z plane and rectangular projections in x-z and x-y planes. Quadrupole defines an
aperture with four cylindrical pole tips.

In general, the physical volumes that scrape the beam are defined independently of the field maps. This is the more
versatile way to do things, but there are some pitfalls which such an implementation introduces. For example, in
hard-edged fields like pillboxes, tracking errors can be introduced when G4MICE steps into the field region. This can
be avoided by adding windows (probably made of vacuum material) to force GEANT4 to stop tracking, make a small step
over the field boundary, and then restart tracking inside the field. However, such details are left for the user to
implement.

\begin{center}
\tablefirsthead{\hline
{\bfseries Volume} &
\multicolumn{1}{m{4.321cm}|}{{\bfseries Dimension1}} &
\multicolumn{1}{m{4.564cm}|}{{\bfseries Dimension2}} &
{\bfseries Dimension3}\\}
\tablehead{\hline
{\bfseries Volume} &
\multicolumn{1}{m{4.321cm}|}{{\bfseries Dimension1}} &
\multicolumn{1}{m{4.564cm}|}{{\bfseries Dimension2}} &
{\bfseries Dimension3}\\}
\tabletail{}
\tablelasttail{}
\begin{supertabular}{|m{2.659cm}|m{4.321cm}m{4.564cm}m{5.25cm}|}
\hline
None &
\multicolumn{3}{m{14.535cm}|}{No dimensions required. Note cannot define daughter Modules for this volume type.}\\\hline
Cylinder &
\multicolumn{1}{m{4.321cm}|}{Radius} &
\multicolumn{1}{m{4.564cm}|}{Length in z} &
Not used (leave blank)\\\hline
Box &
\multicolumn{1}{m{4.321cm}|}{Width in x} &
\multicolumn{1}{m{4.564cm}|}{Height in y} &
Length along z\\\hline
Tube &
\multicolumn{1}{m{4.321cm}|}{Inner Radius} &
\multicolumn{1}{m{4.564cm}|}{Outer Radius} &
Length in z\\\hline
Trapezoid &
\multicolumn{1}{m{4.321cm}|}{Half Width in x} &
\multicolumn{1}{m{4.564cm}|}{Half Height in y} &
Half Length in z\\\hline
Wedge &
\multicolumn{3}{m{14.535cm}|}{See documentation below.}\\\hline
Polycone &
\multicolumn{3}{m{14.535cm}|}{No dimensions required. Volume defined from external file.}\\\hline
Quadrupole &
\multicolumn{3}{m{14.535cm}|}{No dimensions required. Dimensions defined from module properties.}\\\hline
Multipole &
\multicolumn{3}{m{14.535cm}|}{No dimensions required. Dimensions defined from module properties.}\\\hline
Boolean &
\multicolumn{3}{m{14.535cm}|}{No dimensions required. Dimensions defined from module properties.}\\\hline
Sphere &
\multicolumn{3}{m{14.535cm}|}{See documentation below.}\\\hline
\end{supertabular}
\end{center}
\clearpage\subsection{Properties}
Many of the features of G4MICE that can be enabled in a module are described using properties. For example, properties
enable the user to define detectors and fields. Properties can be either of several types: PropertyDouble,
PropertyString, PropertyBool, PropertyHep3Vector or PropertyInt. A property is declared via

{\ttfamily\itshape
\ \ \ \ {\textless}Property Type{\textgreater} {\textless}Property Name{\textgreater} {\textless}Value{\textgreater}
{\textless}Units if appropriate{\textgreater}}

Different properties that can be enabled for Mice Modules are described elsewhere in this document. Properties of type
PropertyDouble and PropertyHep3Vector can have units. Units are defined in the CLHEP library. Units are case sensitive;
G4MICE will return an error message if it fails to parse units. Combinations of units such as T/m or N*m can be defined
using '*' and '/' as appropriate. Properties cannot be defined more than once within the same module.

\subsection{Child Modules}
Child Modules are defined with a position, rotation and scale factor. This places, and rotates, the child volume and any
fields present relative to the parent volume. Scale factor scales fields defined in the child module and any of its
children. Scale factors are recursively multiplicative; that is the field generated by a child module will be scaled by
the product of the scale factor defined in the parent module and all of its parents.

The child module definition looks like:

{\ttfamily
\ \ Module \textit{{\textless}Module File Name{\textgreater}}}

{\ttfamily
\ \ \{}

{\ttfamily
\ \ \ \ Position\ \ \textit{{\textless}x position{\textgreater} {\textless}y position{\textgreater} {\textless}z
position{\textgreater} {\textless}Units{\textgreater}}}

{\ttfamily\itshape
\ \ \ \ \textup{Rotation\ \ }{\textless}x rotation{\textgreater} {\textless}y rotation{\textgreater} {\textless}z
rotation{\textgreater} {\textless}Units{\textgreater}}

{\ttfamily\itshape
\ \ \ \ \textup{ScaleFactor\ \ }{\textless}Value{\textgreater}}

{\ttfamily\itshape
\ \ \ \ {\textless}Define volume, dimensions and properties for this instance only{\textgreater}}

{\ttfamily
\ \ \}}

\textit{{\textless}Module File Name{\textgreater}} is defined relative to the folder
\textit{\$\{MICEFILES\}/Models/Modules/}. The position and rotation default to 0 0 0 and the scale factor defaults to
1.

\liststyleLi
\begin{itemize}
\item Volume, Dimension and Properties of the child module can be defined at the level of the parent; in this case these
values will be used only for this instance of the module.
\end{itemize}
\liststyleLii
\begin{itemize}
\item If no file can be found, G4MICE will press on regardless, attempting to build a geometry using the information
defined in the parent volume.
\end{itemize}
\subsection[Module Hierarchy and GEANT4 Physical Volumes]{Module Hierarchy and GEANT4 Physical Volumes}
[Warning: Draw object ignored]G4MICE enables users to place arbitrary physical volumes in a GEANT4 geometry. The
formatting of G4MICE is such that users are encouraged to use the GEANT4 tree structure for placing physical volumes.
This is a double-edged sword, in that it provides users with a convenient interface for building geometries, but it is
not a terribly safe interface.

Consider the cartoon of physical volumes shown above. Here there is a blue volume sitting inside a red volume sitting
inside the black world volume. For the volumes to be represented properly, the module that represents the blue volume
MUST be a child of the module that represents the red volume. The module that represents the red volume MUST, in turn,
be a child of the module that represents the black volume, in this case the Configuration file.

What would happen if we placed the blue volume directly into the Black volume, i.e. the Configuration file? GEANT4 would
silently ignore the blue volume, or the red volume, depending on the order in which they are added into the GEANT4
geometry. What would happen if we placed the blue volume overlapping the red and black volumes? The behaviour of GEANT4
is not clear in this case.

\liststyleLiii
\begin{itemize}
\item Never allow a volume to overlap any part of another volume that is not it's direct parent.
\end{itemize}
It is possible to check for overlaps by setting the datacard \textit{CheckVolumeOverlaps} to 1.

\subsection{}
\clearpage\subsection{A Sample Configuration File}
Below is listed a sample configuration file, which is likely to be included in the file
\textit{ExampleConfiguration.dat;} the actual name is specified by the datacard MiceModel.

{\ttfamily
\ \ Configuration ExampleConfiguration}

{\ttfamily
\ \ \{}

{\ttfamily
\ \ \ \ Dimensions 1500.0 1000.0 5000.0 cm}

{\ttfamily
\ \ \ \ PropertyString Material AIR}

{\ttfamily
\ \ \ \ Substitution \ \ \$MyRedColour 0.75}

{\ttfamily
\ \ \ \ Module BeamLine/SolMag.dat}

{\ttfamily
\ \ \ \ \{}

{\ttfamily
\ \ \ \ \ \ Position 140.0 0.0 -2175.0 cm}

{\ttfamily
\ \ \ \ \ \ Rotation 0.0 30.0 0.0 degree}

{\ttfamily
\ \ \ \ \ \ ScaleFactor 1.}

{\ttfamily
\ \ \ \ \}}

{\ttfamily
\ \ \ \ Module BeamLine/BendMag.dat}

{\ttfamily
\ \ \ \ \ \ \{}

{\ttfamily
\ \ \ \ \ \ \ \ Position 0.0 0.0 -1935.0 cm}

{\ttfamily
\ \ \ \ \ \ \ \ Rotation 0.0 15.0 0.0 degree}

{\ttfamily
\ \ \ \ \ \ ScaleFactor 1.}

{\ttfamily
\ \ \ \ \ \ \}}

{\ttfamily
\ \ \ \ Module NoFile\_Box1}

{\ttfamily
\ \ \ \ \ \ \{}

{\ttfamily
\ \ \ \ \ \ Volume \ \ \ Box}

{\ttfamily
\ \ \ \ \ \ Dimension 1.0 1.0 1.0}

{\ttfamily
\ \ \ \ \ \ \ \ Position 0.0 0.0 \ 200.0 cm}

{\ttfamily
\ \ \ \ \ \ \ \ Rotation 0.0 15.0 0.0 degree}

{\ttfamily
\ \ \ \ \ \ PropertyString Material Galactic}

{\ttfamily
\ \ \ \ \ \ PropertyDouble RedColour \$MyRedColour}

{\ttfamily
\ \ \ \ \ \ Module NoFile\_Box2}

{\ttfamily
\ \ \ \ \ \ \ \ \{}

{\ttfamily
\ \ \ \ \ \ \ \ Volume \ \ \ Box}

{\ttfamily
\ \ \ \ \ \ \ \ Dimension 0.5 0.5 0.5*3 m //z length = 0.5*3 = 1.5 m}

{\ttfamily
\ \ \ \ \ \ \ \ \ \ Rotation \ 0.0 15.0 0.0 degree //Rotation relative to parent}

{\ttfamily
\ \ \ \ \ \ \ \ PropertyString Material Galactic}

{\ttfamily
\ \ \ \ \ \ \ \ PropertyDouble RedColour \$MyRedColour}

{\ttfamily
\ \ \ \ \ \ \}}

{\ttfamily
\ \ \ \ \ \ \}}

\ \ \}

\clearpage\subsection{A Sample Child Module File}
Below is listed a sample module file, which is likely to be included in the file \textit{SolMag.dat}; the actual
location is specified by the module or configuration that calls FCoil. The module contains a number of properties that
define the field.

{\ttfamily
\ \ Module SolMag}

{\ttfamily
\ \ \{}

{\ttfamily
\ \ \ \ Volume Tube }

{\ttfamily
\ \ \ \ Dimensions 263.0 347.0 210.0 mm}

{\ttfamily
\ \ \ \ PropertyString Material Al}

{\ttfamily
\ \ \ \ PropertyDouble BlueColour 0.75}

{\ttfamily
\ \ \ \ PropertyDouble GreenColour 0.75}

{\ttfamily
\ \ \ \ //field}

{\ttfamily
\ \ \ \ PropertyString FieldType \ \ \ \ \ \ Solenoid}

{\ttfamily
\ \ \ \ PropertyString FileName \ \ \ \ \ \ \ focus.dat}

{\ttfamily
\ \ \ \ PropertyDouble CurrentDensity \ \ \ \ \ 1.}

{\ttfamily
\ \ \ \ PropertyDouble Length \ \ \ \ \ \ \ \ \ \ \ 210. mm}

{\ttfamily
\ \ \ \ PropertyDouble Thickness \ \ \ \ \ \ \ \ \ 84. mm}

{\ttfamily
\ \ \ \ PropertyDouble InnerRadius \ \ \ \ \ \ 263. mm}

{\ttfamily
\ \ \}}

\subsection{}
\clearpage\section{Geometry and Tracking Properties}
Properties for various aspects of the physical and engineering model of the simulation are described below. This
includes properties for sensitive detectors.

\subsection{General Properties}
There are a number of properties that are applicable to any MiceModule.

\begin{center}
\tablefirsthead{\hline
{\bfseries Property} &
{\bfseries Type} &
{\bfseries Description}\\}
\tablehead{\hline
{\bfseries Property} &
{\bfseries Type} &
{\bfseries Description}\\}
\tabletail{}
\tablelasttail{}
\begin{supertabular}{|m{3.665cm}|m{1.278cm}|m{12.052cm}|}
\hline
Material &
string &
The material that the volume is made up from\\\hline
Invisible &
bool &
Set to 1 to make the object invisible in visualisation, or 0 to make the object visible.\\\hline
\multicolumn{2}{m{5.143cm}|}{\hspace*{-\tabcolsep}\begin{tabular}{|m{3.665cm}|m{1.278cm}}

RedColour &
double\\\hline
GreenColour &
double\\\hline
BlueColour &
double\\\hline
\end{tabular}\hspace*{-\tabcolsep}
} &
Alter the colour of objects as they are visualised\\\hhline{~~-}
G4StepMax &
double &
The maximum step length that Geant4 can make in the volume. Inherits values from the parent volumes.\\\hline
\multicolumn{2}{m{5.143cm}|}{\hspace*{-\tabcolsep}\begin{tabular}{|m{3.665cm}|m{1.278cm}}

G4TrackMax &
double\\\hline
G4TimeMax &
double\\\hline
\end{tabular}\hspace*{-\tabcolsep}
} &
The maximum track length and particle time of a track. Tracks outside this bound are killed. Inherits values from the
parent volumes.\\\hhline{~~-}
G4KinMin &
double &
The minimum kinetic energy of a track. Tracks outside this bound are killed. Inherits values from the parent
volumes.\\\hline
SensitiveDetector &
string &
Set to the type of sensitive detector required. Possible sensitive detectors are \textit{TOF}, \textit{SciFi},
\textit{CKOV}, \textit{SpecialVirtual, Virtual, Envelope} or \textit{EMCAL}.\\\hline
\end{supertabular}
\end{center}
\subsection{Sensitive Detectors}
A sensitive detector (one in which hits are recorded) can be defined by including the SensitiveDetector property. When a
volume is set to be a sensitive detector G4MICE will automatically record tracks entering, exiting and crossing the
volume. Details such as the energy deposited by the track are sometimes also recorded in order to enable subsequent
modelling of the detector response.

Some sensitive detectors use extra properties.

\subsubsection[Scintillating Fibre Detector (SciFi)]{Scintillating Fibre Detector (SciFi)}
\subsubsection{Cerenkov Detector (CKOV)}
\subsubsection{Time Of Flight Counter (TOF)}
\subsubsection{Special Virtual Detectors}
Special virtual detectors are used to monitor tracking through a particular physical volume. Normally particle tracks
are written in the global coordinate system, although an alternate coordinate system can be defined. Additional
properties can be used to parameterise special virtual detectors.

\begin{center}
\tablefirsthead{\hline
{\bfseries Property} &
{\bfseries Type} &
{\bfseries Description}\\}
\tablehead{\hline
{\bfseries Property} &
{\bfseries Type} &
{\bfseries Description}\\}
\tabletail{}
\tablelasttail{}
\begin{supertabular}{|m{3.665cm}|m{1.3579999cm}|m{11.973001cm}|}
\hline
\multicolumn{2}{m{5.223cm}|}{\hspace*{-\tabcolsep}\begin{tabular}{|m{3.665cm}|m{1.3579999cm}}

ZSegmentation &
int\\\hline
PhiSegmentation &
int\\\hline
RSegmentation &
int\\\hline
\end{tabular}\hspace*{-\tabcolsep}
} &
Set the number of segments in the detector in Z, R or f. Defaults to 1.\\\hhline{~~-}
\multicolumn{2}{m{5.223cm}|}{\hspace*{-\tabcolsep}\begin{tabular}{|m{3.665cm}|m{1.3579999cm}}

SteppingThrough &
bool\\\hline
SteppingInto &
bool\\\hline
SteppingOutOf &
bool\\\hline
SteppingAcross &
bool\\\hline
\end{tabular}\hspace*{-\tabcolsep}
} &
Set to true to record tracks stepping through, into, out of or across the volume. Defaults to true.\\\hhline{~~-}
Station &
int &
Define an integer that is written to the output file to identify the station. Defaults to a unique integer identifier
chosen by G4MICE, which will be different each time the same Special Virtual is placed.\\\hline
LocalRefRotation &
Hep3

Vector &
If set, record hits relative to a reference rotation in the coordinate system of the SpecialVirtual detector.\\\hline
GlobalRefRotation &
Hep3

Vector &
If set, record hits relative to a reference rotation in the coordinate system of the Configuration.\\\hline
LocalRefPosition &
Hep3

Vector &
If set, record hits relative to a reference position in the coordinate system of the SpecialVirtual detector.\\\hline
GlobalRefPosition &
Hep3

Vector &
If set, record hits relative to a reference position in the coordinate system of the Configuration.\\\hline
\end{supertabular}
\end{center}
\subsubsection{}
\clearpage\subsubsection{Virtual Detectors}
Virtual detectors are used to extract all particle data at a particular plane, irrespective of geometry. Virtual
detectors do not need to have a physical volume. The \textit{plane} can be a plane in z, time, proper time, or a
physical plane with some arbitrary rotation and translation.

\begin{center}
\tablefirsthead{\hline
{\bfseries Property} &
{\bfseries Type} &
{\bfseries Description}\\}
\tablehead{\hline
{\bfseries Property} &
{\bfseries Type} &
{\bfseries Description}\\}
\tabletail{}
\tablelasttail{}
\begin{supertabular}{|m{3.665cm}|m{1.3579999cm}|m{11.973001cm}|}
\hline
{\itshape IndependentVariable} &
String &
\liststyleLiv
\begin{itemize}
\item If set to \textit{t}, particle data will be written for particles at the time defined by the \textit{PlaneTime}
property. 
\end{itemize}
\liststyleLv
\begin{itemize}
\item If set to \textit{tau},\textit{ }particle data will be written for particles at the proper time defined by the
PlaneTime property. 
\end{itemize}
\liststyleLvi
\begin{itemize}
\item If set to \textit{z}, particle data will be written for particles crossing the module's z-position. 
\end{itemize}
\liststyleLvii
\begin{itemize}
\item If set to \textit{u}, particle data will be written for particles crossing a plane extending in \textit{x} and
\textit{y}.
\end{itemize}
\\\hline
{\itshape PlaneTime} &
Double &
If \textit{IndependentVariable} is \textit{t }or \textit{tau,} particle data will be written out at this time. Mandatory
if \textit{IndependentVariable} is \textit{t} or \textit{tau}.\\\hline
RadialExtent &
Double &
If set, particles outside this radius in the plane of the detector will not be recorded by the Virtual detector.\\\hline
GlobalCoordinates &
Bool &
If set to 0, particle data is written in the coordinate system of the module. Otherwise particle data is written in
global coordinates.\\\hline
MultiplePasses &
String &
Set how the VirtualPlane handles particles that pass through more than once. If set to Ignore, particles will be ignored
on second and subsequent passes. If set to SameStation, particles will be registered with the same station number. If
set to NewStation, particles will be registered with a NewStation number given by the \textit{(total number of
stations) + (this plane's station number)}, i.e. a new station number appropriate for a ring geometry.\\\hline
AllowBackwards &
Bool &
Set to false to prevent backwards-going particles from being recorded. Default is true.\\\hline
\end{supertabular}
\end{center}
\subsubsection{}
\clearpage\subsubsection{Envelope Detectors}
Envelope detectors are a type of Virtual detector that take all of the properties listed under virtual detectors, above.
In addition, in the optics application they can be used to interact with the beam envelope in a special way. The
following properties can be defined for Envelope Detectors \textit{in addition to} the properties specified above for
virtual detectors.

The The EnvelopeOut properties are used to make output from the envelope for use in the Optics optimiser.

\begin{center}
\tablefirsthead{\hline
{\bfseries Property} &
{\bfseries Type} &
{\bfseries Description}\\}
\tablehead{\hline
{\bfseries Property} &
{\bfseries Type} &
{\bfseries Description}\\}
\tabletail{}
\tablelasttail{}
\begin{supertabular}{|m{3.665cm}|m{1.3579999cm}|m{11.973001cm}|}
\hline
{\itshape EnvelopeOut1\_Name} &
String &
Defines the variable name that can be used as an expression substitution at the end of each iteration, typically
substituted into the Score parameters in the optimiser (see optimiser, below).\\\hline
{\itshape EnvelopeOut1\_Type} &
String &
Defines the type of variable that will be calculated for the substitution. Options are

\liststyleLviii
\begin{itemize}
\item Mean
\item Covariance
\item Standard\_Deviation
\item Correlation
\item Bunch\_Parameter
\end{itemize}
\\\hline
{\itshape EnvelopeOut1\_Variable} &
String &
Defines the variable that will be calculated for the substitution. Options are for Bunch\_Parameter

\liststyleLix
\begin{itemize}
\item \begin{itemize}
\item {\itshape emit\_6d \textup{: 6d emittance}}
\item {\itshape emit\_4d:\textup{ 4d emittance (in x-y space)}}
\item {\itshape emit\_t:\textup{ 2d emittance (in time space)}}
\item {\itshape emit\_x:\textup{ 2d emittance (in x space)}}
\item {\itshape emit\_y:\textup{ 2d emittance (in y space)}}
\item {\itshape beta\_4d:\textup{ 4d transverse beta function}}
\item {\itshape beta\_t:\textup{ 2d longitudinal beta function}}
\item {\itshape beta\_x: \textup{2d beta function (in(x space)}}
\item {\itshape beta\_y: \textup{2d beta function (in y space)}}
\item {\itshape alpha\_4d:\textup{ 4d transverse alpha function}}
\item {\itshape alpha\_t:\textup{ 2d longitudinal alpha function}}
\item {\itshape alpha\_x: \textup{2d alpha function (in(x space)}}
\item {\itshape alpha\_y: \textup{2d alpha function (in y space)}}
\item {\itshape gamma\_4d:\textup{ 4d transverse gamma function}}
\item {\itshape gamma\_t:\textup{ 2d longitudinal gamma function}}
\item {\itshape gamma\_x: \textup{2d gamma function (in(x space)}}
\item {\itshape gamma\_y: \textup{2d gamma function (in y space)}}
\item {\itshape disp\_x:\textup{ x-dispersion}}
\item {\itshape disp\_y:\textup{ y-dispersion}}
\item {\itshape ltwiddle:\textup{ normalised angular momentum}}
\item {\itshape lkin:\textup{ standard angular momentum}}
\end{itemize}

\end{itemize}
For Mean, Standard\_Deviation, Covariance and Correlation, variables should be selected from the options

\liststyleLx
\begin{itemize}
\item \textit{x: x-}position
\item {\itshape y:y-position}
\item {\itshape t: \textup{time}}
\item {\itshape px:\textup{ x-momentum}}
\item {\itshape py:\textup{ y-momentum}}
\item {\itshape E:\textup{ energy}}
\end{itemize}
For Mean, a single variable should be selected and value corresponding to the reference trajectory will be returned.

For Standard\_Deviation, a single variable should be selected and the 1 sigma beam size will be returned.

For Covariance and Correlation, two variables should be selected separated by a comma.\\\hline
\end{supertabular}
\end{center}
\subsection{Unconventional Volumes}
It is possible to define a number of volumes that use properties rather than the Dimensions keyword to define the volume
size.

{\sffamily\bfseries
Volume Trapezoid}

Volume Trapezoid gives a trapezoid which is not necessarily isosceles. Its dimensions are given by:

\begin{center}
\tablefirsthead{\hline
{\bfseries Property} &
{\bfseries Type} &
{\bfseries Description}\\}
\tablehead{\hline
{\bfseries Property} &
{\bfseries Type} &
{\bfseries Description}\\}
\tabletail{}
\tablelasttail{}
\begin{supertabular}{|m{3.665cm}|m{1.278cm}|m{12.052cm}|}
\hline
TrapezoidWidthX1 &
Double &
Gives width1 in x\\\hline
TrapezoidWidthX2 &
Double &
Gives width2 in x\\\hline
TrapezoidWidthY1 &
Double &
Gives height1 in y\\\hline
TrapezoidWidthY2 &
Double &
Gives height2 in y\\\hline
TrapezoidLengthZ &
Double &
Gives length along z\\\hline
\end{supertabular}
\end{center}
\subsubsection{A Trapezoid Volume is like a Wedge Volume (look visualization below) with the possibility to have
different values for x width and 2 (non-zero) values for y.}
\subsubsection{}
\clearpage\subsubsection{Volume Wedge}
A wedge is a triangular prism as shown in the diagram. Here the blue line extends along the positive z-axis and the red
line extends along the x-axis.

\begin{center}
\includegraphics[width=13.968cm,height=8.394cm]{MiceModule-img001.png}
\end{center}
\begin{center}
\tablefirsthead{\hline
{\bfseries Property} &
{\bfseries Type} &
{\bfseries Description}\\}
\tablehead{\hline
{\bfseries Property} &
{\bfseries Type} &
{\bfseries Description}\\}
\tabletail{}
\tablelasttail{}
\begin{supertabular}{|m{3.665cm}|m{1.278cm}|m{12.052cm}|}
\hline
Dimensions &
Hep3

Vector &
\liststyleLxi
\begin{enumerate}
\item Width of the prism in x
\item Open end height of the prism in y
\item Length of the prism in z
\end{enumerate}
\\\hline
\end{supertabular}
\end{center}
\subsubsection{Volume Polycone}
A polycone is a volume of rotation, defined by a number of points in r and z. The volume is found by a linear
interpolation of the points.

\begin{center}
\tablefirsthead{\hline
{\bfseries Property} &
{\bfseries Type} &
{\bfseries Description}\\}
\tablehead{\hline
{\bfseries Property} &
{\bfseries Type} &
{\bfseries Description}\\}
\tabletail{}
\tablelasttail{}
\begin{supertabular}{|m{3.665cm}|m{1.278cm}|m{12.052cm}|}
\hline
PolyconeType &
string &
Set to Fill to define a solid volume of rotation. Set to Cone to define a shell volume of rotation with an inner and
outer surface.\\\hline
FieldMapMode &
string &
The name of the file that contains the polycone data.\\\hline
\end{supertabular}
\end{center}
\subsubsection{Volume Quadrupole}
Quadrupoles are defined by an empty cylinder with four further cylinders that are approximations to pole tips.

\begin{center}
\tablefirsthead{\hline
{\bfseries Property} &
{\bfseries Type} &
{\bfseries Description}\\}
\tablehead{\hline
{\bfseries Property} &
{\bfseries Type} &
{\bfseries Description}\\}
\tabletail{}
\tablelasttail{}
\begin{supertabular}{|m{3.665cm}|m{1.278cm}|m{12.052cm}|}
\hline
PhysicalLength &
double &
The length of the quadrupole container.\\\hline
QuadRadius &
double &
The distance from the quad centre to the outside of the quad.\\\hline
PoleTipRadius &
double &
The distance from the quad centre to the pole tip.\\\hline
CoilRadius &
double &
~
\\\hline
CoilHalfWidth &
double &
~
\\\hline
BeamlineMaterial &
string &
The material from which the beamline volume is made.\\\hline
QuadMaterial &
string &
The material from which the quadrupole volume is made.\\\hline
\end{supertabular}
\end{center}
\subsubsection{Volume Multipole}
Multipoles are defined by an empty box with an arbitrary number of cylinders that are approximations to pole tips. Poles
are placed around the centre of the box with n-fold symmetry. Multipoles can be curved, in which case poles cannot be
defined -- only a curved rectangular aperture will be present.

\begin{center}
\tablefirsthead{\hline
{\bfseries Property} &
{\bfseries Type} &
{\bfseries Description}\\}
\tablehead{\hline
{\bfseries Property} &
{\bfseries Type} &
{\bfseries Description}\\}
\tabletail{}
\tablelasttail{}
\begin{supertabular}{|m{3.892cm}|m{1.333cm}|m{11.77cm}|}
\hline
{\itshape ApertureCurvature} &
double &
Radius of curvature of the multipole aperture. For now curved apertures cannot have poles. Set to 0 for a straight
aperture.\\\hline
{\itshape ApertureLength} &
double &
Length of the multipole aperture.\\\hline
NumberOfPoles &
int &
Number of poles.\\\hline
PoleCentreRadius &
double &
The distance from the centre of the aperture to the centre of the cylindrical pole.\\\hline
PoleTipRadius &
double &
The distance from the centre of the aperture to the tip of the cylindrical pole.\\\hline
{\itshape ApertureInnerHeight} &
double &
The inner full height of the aperture.\\\hline
{\itshape ApertureInnerWidth} &
double &
The inner full width of the aperture.\\\hline
{\itshape AppertureOuterHeight} &
double &
The outer full height of the aperture.\\\hline
{\itshape ApertureOuterWidth} &
double &
The outer full width of the aperture.\\\hline
\end{supertabular}
\end{center}
\subsubsection{Volume Boolean}
Boolean volumes enable several volumes to be combined to make very sophisticated shapes from a number of elements.
Elements can be combined either by union, intersection or subtraction operations. A union creates a volume that is the
sum of two elements; an intersection creates a volume that covers the region where two volumes intersect each other;
and a subtraction creates a volume that contains all of one volume except the region that another volume sits in.

Boolean volumes combine volumes modelled by other MiceModules (submodules), controlled using the properties listed
below. Only the volume shape is used; position, rotation and field models etc are ignored. Materials, colours and other
relevant properties are all taken only from the Boolean Volume's properties.

Note that unlike in other parts of G4MICE, submodules for use in Booleans (BaseModule, BooleanModule1, BooleanModule2
...) must be defined in a separate file, either defined in \$MICEFILES/Models/Modules or in the working directory.

Also note that visualisation of boolean volumes is rather unreliable. Unfortunately this is a feature of GEANT4. An
alternative technique is to use special virtual detectors to examine hits in boolean volumes.

\begin{center}
\tablefirsthead{\hline
{\bfseries Property} &
{\bfseries Type} &
{\bfseries Description}\\}
\tablehead{\hline
{\bfseries Property} &
{\bfseries Type} &
{\bfseries Description}\\}
\tabletail{}
\tablelasttail{}
\begin{supertabular}{|m{3.89cm}|m{1.333cm}|m{11.772cm}|}
\hline
{\itshape BaseModule} &
string &
Name of the physical volume that the BooleanVolume is based on. This volume will be placed at (0,0,0) with no rotation,
and all subsequent volumes will be added, subtracted or intersected with this one.\\\hline
{\itshape BooleanModule1} &
string &
The first module to add. G4MICE will search for the MiceModule with path
\$MICEFILES/Models/Modules/{\textless}BooleanModule1{\textgreater}.\\\hline
{\itshape BooleanModule1Type} &
string &
The type of boolean operation to apply, either ``Union'', ``Intersection'' or ``Subtraction''.\\\hline
{\itshape BooleanModule1Pos} &
Hep3

Vector &
The position of the new volume with respect to the Base volume.\\\hline
{\itshape BooleanModule1Rot} &
Hep3

Vector &
The rotation of the new volume with respect to the Base volume.\\\hline
{\itshape BooleanModuleN} &
string &
Add extra modules as required. Replace ``N'' with the module number. N must be a continuous series incrementing by 1 for
each new module. Note that the order in which modules are added is important -- (A-B) \textsf{U }C is different to A-(B
\textsf{U }C).\\\hline
{\itshape BooleanModuleNType} &
\multicolumn{1}{m{1.333cm}}{string} &
\\\hhline{--~}
{\itshape BooleanModuleNPos} &
\multicolumn{1}{m{1.333cm}}{Hep3

Vector} &
\\\hhline{--~}
{\itshape BooleanModuleNRot} &
\multicolumn{1}{m{1.333cm}}{Hep3

Vector} &
\\\hhline{--~}
\end{supertabular}
\end{center}
\subsubsection{Volume Sphere}
A sphere is a spherical shell, with options for opening angles to make segments.

\begin{center}
\tablefirsthead{\hline
{\bfseries Property} &
{\bfseries Type} &
{\bfseries Description}\\}
\tablehead{\hline
{\bfseries Property} &
{\bfseries Type} &
{\bfseries Description}\\}
\tabletail{}
\tablelasttail{}
\begin{supertabular}{|m{3.89cm}|m{1.333cm}|m{11.772cm}|}
\hline
{\itshape Dimensions} &
Hep3

Vector &
The x value defines the inner radius. The y value defines the outer radius of the shell. The z value is not
used.\\\hline
Phi &
Hep3

Vector &
The x value defines the start opening angle in phi. The y value defines the end opening angle. The z value is not used.
Phi values must be in the range 0 to 360 degrees. If undefined, defaults to the range 0-360 degrees.\\\hline
Theta &
Hep3

Vector &
The x value defines the start opening angle in theta. The y value defines the end opening angle. The z value is not
used. Theta values must be in the range 0 to 180 degrees. If undefined, defaults to the range 0-360 degrees.\\\hline
\end{supertabular}
\end{center}
\clearpage\subsection{Repeating Modules}
It is possible to set up a repeating structure for e.g. a repeating magnet lattice. The RepeatModule property enables
the user to specify that a particular module will be repeated a number of times, with all properties passed onto the
child module, but with a new position, orientation and scale factor. Each successive repetition will be given a
translation and a rotation relative to the coordinate system of the previous repetition, enabling the construction of
circular and straight accelerator lattices. Additionally, successive repetitions can have fields scaled relative to
previous repetitions, enabling for example alternating lattices.

\begin{center}
\tablefirsthead{\hline
{\bfseries Property} &
{\bfseries Type} &
{\bfseries Description}\\}
\tablehead{\hline
{\bfseries Property} &
{\bfseries Type} &
{\bfseries Description}\\}
\tabletail{}
\tablelasttail{}
\begin{supertabular}{|m{3.892cm}|m{1.333cm}|m{11.77cm}|}
\hline
{\itshape RepeatModule} &
bool &
Set to 1 to enable repeats in this module.\\\hline
{\itshape NumberOfRepeats} &
int &
Number of times the module will be repeated in addition to the initial placement.\\\hline
{\itshape RepeatTranslation} &
Hep3

Vector &
Translation applied to successive repeats, applied in the coordinate system of the previous repetition.\\\hline
{\itshape RepeatRotation} &
Hep3

Vector &
Rotation applied to successive repeats, applied in the coordinate system of the previous repetition.\\\hline
{\itshape RepeatScaleFactor} &
double &
ScaleFactor applied to successive repeats, applied relative to previous repetition's scale factor.\\\hline
\end{supertabular}
\end{center}
The RepeatModule2 property also enables the user to specify that a particular module will be repeated a number of times.
In this case, G4MICE will set a substitution variable @RepeatNumber that holds an index between 0 and NumberOfRepeats.
This can be used in an expression in to provide a versatile interface between user and accelerator lattice.

\begin{center}
\tablefirsthead{\hline
{\bfseries Property} &
{\bfseries Type} &
{\bfseries Description}\\}
\tablehead{\hline
{\bfseries Property} &
{\bfseries Type} &
{\bfseries Description}\\}
\tabletail{}
\tablelasttail{}
\begin{supertabular}{|m{3.892cm}|m{1.333cm}|m{11.77cm}|}
\hline
{\itshape RepeatModule2} &
bool &
Set to 1 to enable repeats in this module.\\\hline
{\itshape NumberOfRepeats} &
int &
Number of times the module will be repeated in addition to the initial placement.\\\hline
\end{supertabular}
\end{center}
\subsection{}
\clearpage\subsection{Beam Definition and Beam Envelopes}
The Optics application can be used to track a trajectory and associated beam envelope through the accelerator structure.
Optics works by finding the Jacobian around some arbitrary trajectory using a numerical differentiation. This is used
to define a linear mapping about this trajectory, which can then be used to transport the beam envelope.

The Simulation application can be used to generate a random particle beam with a Gaussian distribution and RMS
parameters defined in the same way as the Optics beam envelope. Alternatively, pencil beams and beams from some input
file can also be defined here.

A beam envelope is defined by a reference trajectory and a beam ellipse. The reference trajectory takes its position and
direction from the position and rotation of the module. No rotation builds a reference trajectory along the z-axis. The
magnitude of the momentum and the initial time of the reference trajectory is defined by properties. RF cavities are
phased using the reference trajectory defined here.

The beam ellipse is represented by a matrix, which can either be set using 

\liststyleLxii
\begin{itemize}
\item Twiss-style parameters in (x,px),(y,py) and (t,E) spaces.
\item Twiss-style parameters in (t,E) space and Penn-style parameters in a cylindrically symmetric (x,px,y,py) space.
\item A 6x6 beam ellipse matrix where the ellipse equation is given by \textbf{X}.\textbf{T}() \textbf{M} \textbf{X} =
1.
\end{itemize}
The Penn ellipse matrix is given by

\begin{equation*}
M=\left(\begin{matrix}\epsilon _Lmc\frac{\beta _L}{p}&-\epsilon _Lmc\alpha _L&0&0&0&0\\&\epsilon _Lmc\gamma
_Lp&\frac{D_x}{E}V(E)&\frac{D_x'}{E}V(E)&\frac{D_y}{E}V(E)&\frac{D_y'}{E}V(E)\\&&\epsilon _Tmc\frac{\beta _T}{p}&-\epsilon
_Tmc\alpha _T&0&-\epsilon _Tmc(\frac{q}{2}\beta _T\frac{B_z}{P}-L)\\&&&\epsilon _Tmc\gamma _Tp&\epsilon _Tmc(\frac{q}{
2}\beta _T\frac{B_z}{P}-L)&0\\&&&&\epsilon _Tmc\frac{\beta _T}{p}&-\epsilon _Tmc\alpha _T\\&&&&&\epsilon _Lmc\gamma
_Tp\end{matrix}\right)
\end{equation*}
Here L is a normalised canonical angular momentum, \textit{q} is the reference particle charge,
\textit{B}\textit{\textsubscript{z}} is the nominal on-axis magnetic field, \textit{p} is the reference momentum,
\textit{m} is the reference mass, e\textsubscript{T} is the transverse emittance, b\textsubscript{T} and
a\textsubscript{T} are the transverse Twiss-like functions, e\textsubscript{L} is the longitudinal emittance and
b\textsubscript{L} and a\textsubscript{L} are the longitudinal Twiss-like functions. Additionally D\textsubscript{x},
D\textsubscript{y}, D'\textsubscript{x}, D'\textsubscript{y} are the dispersions and their derivatives with respect to
z and V(E) is the variance of energy (given by the (2,2) term in the matrix above).

The Twiss ellipse matrix is given by

\begin{equation*}
M=\left(\begin{matrix}\epsilon _Lmc\frac{\beta _L}{p}&-\epsilon _Lmc\alpha _L&0&0&0&0\\&\epsilon _Lmc\gamma
_Lp&\frac{D_x}{E}V(E)&\frac{D_x'}{E}V(E)&\frac{D_y}{E}V(E)&\frac{D_y'}{E}V(E)\\&&\epsilon _xmc\frac{\beta _x}{p}&-\epsilon
_xmc\alpha _x&0&0\\&&&\epsilon _xmc\gamma _xp&0&0\\&&&&\epsilon _ymc\frac{\beta _y}{p}&-\epsilon _ymc\alpha
_y\\&&&&&\epsilon _ymc\gamma _yp\end{matrix}\right)
\end{equation*}
Here \textit{p} is the reference momentum, \textit{m} is the reference mass, e\textsubscript{i}, b\textsubscript{i} and
a\textsubscript{i} are the emittances and Twiss functions in the (t,E), (x,p\textsubscript{x}) and
(y,p\textsubscript{y}) planes respectively, D\textsubscript{x}, D\textsubscript{y}, D'\textsubscript{x},
D'\textsubscript{y} are the dispersions and their derivatives with respect to z and V(E) is the variance of energy
(given by the (2,2) term in the matrix above).

\begin{center}
\tablefirsthead{\hline
{\bfseries Property} &
{\bfseries Type} &
{\bfseries Description}\\}
\tablehead{\hline
{\bfseries Property} &
{\bfseries Type} &
{\bfseries Description}\\}
\tabletail{}
\tablelasttail{}
\begin{supertabular}{|m{3.319cm}|m{1.2019999cm}|m{11.974cm}|}
\hline
{\itshape EnvelopeType} &
string &
Set to \textit{TrackingDerivative} to evolve a beam envelope in the Optics application.\\\hline
{\itshape BeamType} &
string &
Set to \textit{Random} to generate a beam using the parameters below for the Simulation application. Set to
\textit{Pencil }to generate a pencil beam (with no random distribution). Set to \textit{ICOOL}, \textit{Turtle,
G4MICE\_PrimaryGenHit} or \textit{G4BeamLine} to use a beam file.\\\hline
{\itshape Pid} &
int &
The particle ID of particles in the envelope or beam.\\\hline
{\itshape Longitudinal}

{\itshape Variable} &
string &
Set the longitudinal variable used to define the reference trajectory momentum. Options are \textit{Energy},
\textit{KineticEnergy, Momentum} and\textit{ ZMomentum.}\\\hline
\multicolumn{1}{m{3.319cm}}{\hspace*{-\tabcolsep}\begin{tabular}{|m{3.319cm}}

Energy\\\hline
KineticEnergy\\\hline
Momentum\\\hline
ZMomentum\\\hline
\end{tabular}\hspace*{-\tabcolsep}
} &
\hspace*{-\tabcolsep}\begin{tabular}{|m{1.2019999cm}}

double\\\hline
double\\\hline
double\\\hline
double\\\hline
\end{tabular}\hspace*{-\tabcolsep}
 &
Define the value of the longitudinal variable used to calculate the mean momentum and energy. The usual relationship
E\textsuperscript{2}+p\textsuperscript{2}c\textsuperscript{2}=m\textsuperscript{2}c\textsuperscript{4} applies. Kinetic
energy E\textsubscript{k} is related to energy E by E\textsubscript{k}+m=E.\\\hhline{~~-}
{\itshape EllipseDefinition} &
string &
Define the beam ellipse that will be used in calculating the evolution of the Envelope, or used to generate a beam for
BeamType \textit{Random}. Options are \textit{Twiss, Penn} and\textit{ Matrix}.\\\hline
\multicolumn{3}{|m{16.894999cm}|}{{\itshape The following properties are only used if EllipseDefinition is set to
Twiss}}\\\hline
\multicolumn{2}{m{4.721cm}|}{\hspace*{-\tabcolsep}\begin{tabular}{|m{3.319cm}|m{1.2019999cm}}

{\itshape Emittance\_X} &
double\\\hline
{\itshape Emittance\_Y} &
double\\\hline
{\itshape Emittance\_L} &
double\\\hline
\end{tabular}\hspace*{-\tabcolsep}
} &
Emittance in each 2d subspace, (x,px), (y,py) and (t,E).\\\hhline{~~-}
\multicolumn{2}{m{4.721cm}|}{\hspace*{-\tabcolsep}\begin{tabular}{|m{3.319cm}|m{1.2019999cm}}

{\itshape Beta\_X} &
double\\\hline
{\itshape Beta\_Y} &
double\\\hline
{\itshape Beta\_L} &
double\\\hline
\end{tabular}\hspace*{-\tabcolsep}
} &
Twiss b function in each 2d subspace, (x,px), (y,py) and (t,E).\\\hhline{~~-}
\multicolumn{2}{m{4.721cm}|}{\hspace*{-\tabcolsep}\begin{tabular}{|m{3.319cm}|m{1.2019999cm}}

{\itshape Alpha\_X} &
double\\\hline
{\itshape Alpha\_Y} &
double\\\hline
{\itshape Alpha\_L} &
double\\\hline
\end{tabular}\hspace*{-\tabcolsep}
} &
Twiss a function in each 2d subspace, (x,px), (y,py) and (t,E).\\\hhline{~~-}
\multicolumn{3}{|m{16.894999cm}|}{{\itshape The following properties are only used if EllipseDefinition is set to
Matrix}}\\\hline
\multicolumn{2}{m{4.721cm}|}{\hspace*{-\tabcolsep}\begin{tabular}{|m{3.319cm}|m{1.2019999cm}}

{\itshape Covariance(t,t)} &
double\\\hline
{\itshape Covariance(t,E)} &
double\\\hline
{\itshape Covariance(t,x)} &
double\\\hline
{\itshape ...} &
double\\\hline
{\itshape Covariance(Py,Py)} &
double\\\hline
\end{tabular}\hspace*{-\tabcolsep}
} &
Set the 6x6 matrix that will be used in the to define the beam ellipse. Covariances should be covariances of elements of
the matrix (x,Px,y,Py,t,E).

\ This must be a positive definite matrix, i.e. determinant {\textgreater} 0. Note that this means that at least the 6
terms on the diagonal must be defined. Other terms default to 0.\\\hline
\multicolumn{3}{|m{16.894999cm}|}{{\itshape The following properties are only used if EllipseDefinition is set to
Penn}}\\\hline
{\itshape Emittance\_T} &
double &
Transverse emittance for the 4d (x,px,y,py) subspace.\\\hline
{\itshape Emittance\_L} &
double &
Longitudinal emittance for the 2d (t,E) subspace.\\\hline
{\itshape Beta\_T} &
double &
Transverse beta for the 4d (x,px,y,py) subspace.\\\hline
{\itshape Beta\_L} &
double &
Longitudinal beta for the 2d (t,E) subspace.\\\hline
{\itshape Alpha\_T} &
double &
Transverse alpha for the 4d (x,px,y,py) subspace.\\\hline
{\itshape Alpha\_L} &
double &
Longitudinal alpha for the 2d (t,E) subspace.\\\hline
{\itshape Normalised}

{\itshape AngularMomentu} &
double &
Normalised angular momentum for the transverse phase space.\\\hline
Bz &
double &
Nominal magnetic field on the reference particle.\\\hline
\multicolumn{3}{|m{16.894999cm}|}{{\itshape The following properties are used if EllipseDefinition is set to Penn or
Twiss}}\\\hline
Dispersion\_X &
double &
Dispersion in x (x-energy correlation).\\\hline
Dispersion\_Y &
double &
Dispersion in y (y-energy correlation).\\\hline
DispersionPrime\_X &
double &
D' in x (Px-energy correlation).\\\hline
DispersionPrime\_Y &
double &
D' in y (Py-energy correlation).\\\hline
\multicolumn{3}{|m{16.894999cm}|}{{\itshape The following properties are only relevant for generating a beam
envelope}}\\\hline
RootOutput &
string &
Output file name for writing output beam envelope in ROOT binary format.\\\hline
LongTextOutput &
string &
Output file name for writing output beam envelope in string format.\\\hline
ShortTextOutput &
string &
Output file name for writing output beam envelope in string format. This abbreviated output omits some of the fields
that are present in LongTextOutput files.\\\hline
BeamOutput &
string &
If a BeamType is defined, this property controls the file name to which beam data is written.\\\hline
Delta\_t &
double &
Offset in time used for calculating numerical derivatives. Default is 0.1 ns.\\\hline
Delta\_E &
double &
Offset in energy used for calculating numerical derivatives. Default is 1 MeV.\\\hline
Delta\_x &
double &
Offset in x position used for calculating numerical derivatives. Default is 1 mm.\\\hline
Delta\_Px &
double &
Offset in x momentum used for calculating numerical derivatives. Default is 1 MeV/c.\\\hline
Delta\_y &
double &
Offset in y position used for calculating numerical derivatives. Default is 1 mm.\\\hline
Delta\_Py &
double &
Offset in y momentum used for calculating numerical derivatives. Default is 1 MeV/c.\\\hline
\multicolumn{2}{m{4.721cm}|}{\hspace*{-\tabcolsep}\begin{tabular}{|m{3.319cm}|m{1.2019999cm}}

Max\_Delta\_t &
double\\\hline
Max\_Delta\_E &
double\\\hline
Max\_Delta\_x &
double\\\hline
Max\_Delta\_Px &
double\\\hline
Max\_Delta\_y &
double\\\hline
Max\_Delta\_Py &
double\\\hline
\end{tabular}\hspace*{-\tabcolsep}
} &
Maximum offsets when polyfit algorithm is used. In some cases the offset can keep increasing without limit unless these
maximum offsets are defined. Default is no limit.\\\hhline{~~-}
\multicolumn{3}{|m{16.894999cm}|}{{\itshape The following properties are only relevant for generating a particle
beam}}\\\hline
UseAsReference &
Bool &
If set to true and the datacard \textit{FirstParticleIsReference} is set to 0, the first event in the Module will be
used as the reference particle that sets cavity phases. This particle will then have the mean trajectory (i.e. no
gaussian distribution).\\\hline
{\itshape BeamFile} &
string &
If the BeamType is \textit{ICOOL}, \textit{Turtle, G4MICE\_PrimaryGenHit} or \textit{G4BeamLine}, this property defines
the name of the file containing tracks for G4MICE.\\\hline
NumberOfEvents &
int &
Set the maximum number of events to take from this module. If other modules are defined, G4MICE will iterate over the
modules until it the datacard \textit{numEvts} is reached or all modules have been run to \textit{NumberOfEvents}.
Default is for G4MICE to keep tracking from the first module it finds until \textit{numEvts} is reached.\\\hline
\end{supertabular}
\end{center}
\subsection{Optimiser}
It is possible to define an optimiser for use in the Optics application. The optimiser enables the user to vary
parameters in the MiceModule file and try to find some optimum setting. For each value of the parameters, G4MICE Optics
will calculate a score; the optimiser attempts to find a minimum value for this score.

\begin{center}
\tablefirsthead{\hline
{\bfseries Property} &
{\bfseries Type} &
{\bfseries Description}\\}
\tablehead{\hline
{\bfseries Property} &
{\bfseries Type} &
{\bfseries Description}\\}
\tabletail{}
\tablelasttail{}
\begin{supertabular}{|m{3.319cm}|m{1.2019999cm}|m{11.974cm}|}
\hline
{\itshape Optimiser} &
string &
Controls the function used for optimising. For now Minuit is the only available option.\\\hline
{\itshape Algorithm} &
string &
For \textit{Minuit} optimiser, controls the \textit{Minuit} algorithm used. In general Simplex is a good option to use
here. An alternative is Migrad. See Minuit documentation (for example at http://root.cern.ch/root/html/TMinuit.html)
for further information. Minuit attempts to minimise the score function defined by the Score properties.\\\hline
{\itshape NumberOfTries} &
int &
Maximum number of iterations G4MICE will make in order to find the optimum value.\\\hline
{\itshape StartError} &
double &
Guess at the initial error in the score.\\\hline
{\itshape EndError} &
double &
Required final error in the score for the optimisation to converge successfully.\\\hline
RebuildSimulation &
bool &
Set to False to tell G4MICE not to rebuild the simulation on each iteration. This should be used to speed up the
optimiser when a parameter is used that does not change the field maps. Default is true.\\\hline
{\itshape Parameter1\_Start} &
double &
Seed value for the parameter, that is used in the first iteration.\\\hline
{\itshape Parameter1\_Name} &
string &
Name of the parameter. This name is used as an expression substitution variable elsewhere in the code and should start
with @. See Expression Substitutions above for details on usage of expression substitutions.\\\hline
Parameter1\_Delta &
double &
Estimated initial error on the parameter. Default is 1.\\\hline
Parameter1\_Fixed &
bool &
Set to true to fix the parameter (so that it is excluded from the optimisation). Default is false.\\\hline
Parameter1\_Min &
double &
If required, set to the minimum value that the parameter can hold.\\\hline
Parameter1\_Max &
double &
If required, set to the maximum value that the parameter can hold.\\\hline
\multicolumn{2}{m{4.721cm}}{\hspace*{-\tabcolsep}\begin{tabular}{|m{3.319cm}|m{1.2019999cm}}

Parameter2\_Start &
...\\\hline
... &
...\\\hline
Parameter2\_Max &
...\\\hline
{\itshape Score1} &
double\\\hline
Score2 &
...\\\hline
... &
...\\\hline
\end{tabular}\hspace*{-\tabcolsep}
} &
\multicolumn{1}{m{11.974cm}}{\hspace*{-\tabcolsep}\begin{tabular}{|m{11.974cm}|}

Define an arbitrary number of parameters. Parameters must be numbered consecutively, and each parameter must have at
least the start value and name defined.\\\hline
The optimiser will attempt to optimise against a score that is calculated by summing the Score1, Score2,... parameters
on each iteration.\\\hline
\end{tabular}\hspace*{-\tabcolsep}
}\\
\end{supertabular}
\end{center}
\clearpage\section{Field Properties}
Invoke a field using PropertyString FieldType {\textless}fieldtype{\textgreater}. The field will be placed, normally
centred on the MiceModule Position and with the appropriate Rotation. Further options for each field type are specified
below, and relevant datacards are also given. If a fieldtype is specified some other properties must also be specified,
while others may be optional, usually taking their value from defaults specified in the datacards. Only one fieldtype
can be specified per module. However, fields from multiple modules are superimposed, each transformed according to the
MiceModule specification. This enables many different field configurations to be simulated using G4MICE.

To use BeamTools fields, datacard FieldMode Full must be set. This is the default.

\begin{center}
\tablefirsthead{\hline
{\bfseries Property} &
{\bfseries Type} &
{\bfseries Description}\\}
\tablehead{\hline
{\bfseries Property} &
{\bfseries Type} &
{\bfseries Description}\\}
\tabletail{}
\tablelasttail{}
\begin{supertabular}{|m{3.665cm}|m{1.278cm}|m{12.052cm}|}
\hline
{\itshape FieldType} &
string &
Set the field type of the MiceModule.\\\hline
\end{supertabular}
\end{center}
\subsection{FieldType CylindricalField}
Sets a constant magnetic field in a cylindrical region symmetric about the z-axis of the module.

\begin{center}
\tablefirsthead{\hline
{\bfseries Property} &
{\bfseries Type} &
{\bfseries Description}\\}
\tablehead{\hline
{\bfseries Property} &
{\bfseries Type} &
{\bfseries Description}\\}
\tabletail{}
\tablelasttail{}
\begin{supertabular}{|m{3.665cm}|m{1.278cm}|m{12.052cm}|}
\hline
{\itshape ConstantField} &
Hep3

Vector &
The magnetic field that will be placed in the region.\\\hline
\multicolumn{2}{m{5.143cm}|}{\hspace*{-\tabcolsep}\begin{tabular}{|m{3.665cm}|m{1.278cm}}

{\itshape Length} &
double\\\hline
{\itshape Radius} &
double\\\hline
\end{tabular}\hspace*{-\tabcolsep}
} &
The physical extent of the region.\\\hhline{~~-}
\end{supertabular}
\end{center}
\subsection{FieldType RectangularField}
Sets a constant magnetic field in a rectangular region.

\begin{center}
\tablefirsthead{\hline
{\bfseries Property} &
{\bfseries Type} &
{\bfseries Description}\\}
\tablehead{\hline
{\bfseries Property} &
{\bfseries Type} &
{\bfseries Description}\\}
\tabletail{}
\tablelasttail{}
\begin{supertabular}{|m{3.665cm}|m{1.278cm}|m{12.052cm}|}
\hline
{\itshape ConstantField} &
Hep3

Vector &
The magnetic field that will be placed in the region.\\\hline
\multicolumn{2}{m{5.143cm}|}{\hspace*{-\tabcolsep}\begin{tabular}{|m{3.665cm}|m{1.278cm}}

{\itshape Length} &
double\\\hline
{\itshape Width} &
double\\\hline
{\itshape Height} &
double\\\hline
\end{tabular}\hspace*{-\tabcolsep}
} &
The physical extent of the region.\\\hhline{~~-}
\end{supertabular}
\end{center}
\subsection{FieldType Solenoid}
G4MICE simulates solenoids using a series of current sheets. The field for each solenoid is written to a field map on a
rectangular grid and can then be reused. The field from each current sheet is calculated using the formula for current
sheets detailed in MUCOOL Note 281, \textit{Modeling solenoids using coil, sheet and block conductors}.

\begin{center}
\tablefirsthead{\hline
{\bfseries Property} &
{\bfseries Type} &
{\bfseries Description}\\}
\tablehead{\hline
{\bfseries Property} &
{\bfseries Type} &
{\bfseries Description}\\}
\tabletail{}
\tablelasttail{}
\begin{supertabular}{|m{3.665cm}|m{1.278cm}|m{12.052cm}|}
\hline
{\itshape FileName} &
string &
Read or write solenoid data to the filename. If different modules have the same filename, G4MICE assumes they are the
same.\\\hline
FieldMapMode &
string &
If set to Read, G4MICE will attempt to read existing data from the FileName. If set to Write, G4MICE will generate new
data and write it to the FileName. If set to Analytic, G4MICE will calculate fields directly without interpolating. If
set to WriteDynamic acts as in Write except the grid extent and grid spacing at each point is chosen dynamically to
some tolerance defined in the FieldTolerance property. Takes default from datacard SolDataFiles (Write).\\\hline
\multicolumn{2}{m{5.143cm}|}{\hspace*{-\tabcolsep}\begin{tabular}{|m{3.665cm}|m{1.278cm}}

{\itshape Length} &
double\\\hline
{\itshape Thickness} &
double\\\hline
{\itshape InnerRadius} &
double\\\hline
{\itshape CurrentDensity} &
double\\\hline
\end{tabular}\hspace*{-\tabcolsep}
} &
Coil physical parameters. Only used in Write/Analytic mode where they are mandatory.\\\hhline{~~-}
ZExtentFactor &
double &
Field map extends to length + ZExtentFactor*innerRadius in Write mode. Takes default from datacard SolzMapExtendFactor
(10.). Map size is chosen dynamically in WriteDynamic mode.\\\hline
RExtentFactor &
double &
Field map extends to radius RExtentFactor*innerRadius in Write mode. Takes default from datacard SolrMapExtendFactor
(2.018...). Avoid allowing grid nodes to fall on sheets.\\\hline
NumberOfZCoords &
int &
Number of coordinates in \textit{z} in field map grid in Write mode. Takes default from datacard NumberNodesZGrid
(500).\\\hline
NumberOfRCoords &
int &
Number of coordintes in \textit{r} in field map grid in Write mode. Takes default from datacard NumberNodesRGrid
(100).\\\hline
NumberOfSheets &
int &
Number of sheets used to calculate the field. Takes default from datacard DefaultNumberOfSheets (10).\\\hline
{\itshape FieldTolerance } &
double &
Mandatory when FieldMapMode is WriteDynamic. If field map mode is write dynamic, this datacard controls the tolerance on
errors in the field with which the field grid and the grid extent will be chosen. \\\hline
Interpolation

Algorithm &
string &
Choose the interpolation algorithm. Options are BiLinear for a linear interpolation in \textit{r} and \textit{z}, or
LinearCubic \ for a linear interpolation in \textit{r} and a cubic spline in \textit{z}. Default is
LinearCubic.\\\hline
IsAmalgamated &
bool &
Set to 1 to add the coil to CoilAmalgamtion parent field (see below).\\\hline
\end{supertabular}
\end{center}
\subsection{}
\clearpage\subsection{FieldType FieldAmalgamation}
During tracking, G4MICE stores a list of fields and for each one G4MICE checks to see if tracking is performed through a
particular field map's bounding box. This can be slow if a large number of fields are present. One way to speed this up
is to store contributions from many coils in a single CoilAmalgamation. A CoilAmalgamation searches through child
modules for solenoids with PropertyBool IsAmalgamated set to true. If it finds such a coil, it will add the field
generated by the solenoid to its own field map and disable the coil.

\begin{center}
\tablefirsthead{\hline
{\bfseries Property} &
{\bfseries Type} &
{\bfseries Description}\\}
\tablehead{\hline
{\bfseries Property} &
{\bfseries Type} &
{\bfseries Description}\\}
\tabletail{}
\tablelasttail{}
\begin{supertabular}{|m{3.665cm}|m{1.278cm}|m{12.052cm}|}
\hline
{\itshape Length} &
double &
The Length of the field map generated by the CoilAmalgamation.\\\hline
{\itshape RMax} &
double &
The maximum radius of the field map generated by the CoilAmalgamation.\\\hline
Interpolation

Algorithm &
string &
Choose the interpolation algorithm. Options are BiLinear for a linear interpolation in \textit{r} and \textit{z}, or
LinearCubic \ for a linear interpolation in \textit{r} and a cubic spline in \textit{z}. Default is
LinearCubic.\\\hline
\multicolumn{2}{m{5.143cm}|}{\hspace*{-\tabcolsep}\begin{tabular}{|m{3.665cm}|m{1.278cm}}

{\itshape ZStep} &
double\\\hline
{\itshape RStep} &
double\\\hline
\end{tabular}\hspace*{-\tabcolsep}
} &
Step size of the field map generated by the CoilAmalgamation.\\\hhline{~~-}
\end{supertabular}
\end{center}

\bigskip

\clearpage\subsection{FieldType FastSolenoid}
This is an alternative field model for solenoids that uses a power law expansion of the on-axis magnetic field and its
derivatives, and an exponential fall-off for the fringe field (tanh).

\begin{center}
\tablefirsthead{\hline
{\bfseries Property} &
{\bfseries Type} &
{\bfseries Description}\\}
\tablehead{\hline
{\bfseries Property} &
{\bfseries Type} &
{\bfseries Description}\\}
\tabletail{}
\tablelasttail{}
\begin{supertabular}{|m{3.665cm}|m{1.278cm}|m{12.052cm}|}
\hline
{\itshape PeakField} &
double &
Nominal peak field of the solenoid.\\\hline
{\itshape EFoldLength} &
double &
The fall-off length for the fringe field.\\\hline
{\itshape CentreLength} &
double &
Nominal length for the peak field region.\\\hline
{\itshape Order} &
int &
Order to which the field will be calculated.\\\hline
FieldTolerance &
double &
If positive, G4MICE will abort tracking if a particle crosses through a field with estimated error {\textgreater}
FieldTolerance. G4MICE estimates the error as the field value calculated at highest order. Default is -1 (i.e.
inactive).\\\hline
\end{supertabular}
\end{center}
\subsection{Phasing Models}
G4MICE has a number of sophisticated models for phasing RF cavities. These powerful models can make setting up RF
cavities with realistic fields both quick and easy.

When CavityMode is Unphased, G4MICE attempts to phase the cavity itself. When using CavityMode Unphased G4MICE needs to
know when particles enter, cross the middle, and leave cavities. This means that:

\liststyleLxiii
\begin{itemize}
\item The cavity must sit in a rectangular or cylindrical physical volume.
\item No other physical volumes may overlap or sit within the physical volume of the cavity.
\end{itemize}
If these conditions are not met the phasing algorithm is likely to fail.

To phase a cavity, G4MICE builds a volume in the centre of the cavity that is used for phasing and then fires a
reference particle through the system. Stochastic processes are always disabled during this process, while mean energy
loss can be disabled using the datacard ReferenceEnergyLossModel. If a reference particle crosses a plane through the
centre of a cavity, it sets the phase of the cavity to the time at which the particle crosses. 

The field of the cavity during phasing is controlled by the property FieldDuringPhasing. There are four modes:

\liststyleLxiv
\begin{itemize}
\item \textit{None}: Cavity fields are disabled during phasing
\item \textit{Electrostatic}: An electrostatic field with no positional dependence given by
PeakEField*sin(ReferenceParticlePhase) is enabled during phasing.
\item \textit{TimeVarying}: A standard time varying field is applied during phasing, initially with arbitrary phase
relative to the reference particle. G4MICE uses a Newton-Raphson method to find the appropriate reference phase
iteratively, with tolerance set by the datacard PhaseTolerance.
\item \textit{EnergyGainOptimised}: A standard time varying field is applied during phasing, initially with arbitrary
phase and peak field relative to the reference particle. G4MICE uses a 2D Newton-Raphson method to find the appropriate
reference phase and peak field iteratively, so that the reference particle crosses the cavity centre with phase given
by property ReferenceParticlePhase and gains energy over the whole cavity given by property EnergyGain with tolerances
set by the datacards PhaseTolerance and RFDeltaEnergyTolerance.
\end{itemize}
\subsection{Tracking Stability Around RF Cavities}
Usually RF cavities have little or no fringe field, and this can lead to some instability in the tracking algorithms.
When G4MICE makes a step into an RF cavity volume, the tracking algorithms tend to smooth out a field in a non-physical
way. This can be prevented by either (i) making the step size rather small in the RF cavity or (ii) forcing G4MICE to
stop tracking by adding a physical volume at the entrance of the RF cavity (a window, typically made of vacuum). Doing
this should improve stability of tracking.

\subsection{FieldType PillBox}
Sets a PillBox field in a particular region. G4MICE represents pillboxes using a sinusoidally varying TM010 pill box
field, with non-zero field vector elements given by

\begin{equation*}
\begin{gathered}B_{\phi }=J_1(k_rr)\cos (\omega t)\\E_z=J_0(k_rr)\cos (\omega t)\end{gathered}
\end{equation*}
Here J\textsubscript{n} are Bessel functions and k\textsubscript{r} is a constant. See, for example, SY Lee VI.1. All
other fields are 0.

\begin{center}
\tablefirsthead{\hline
{\bfseries Property} &
{\bfseries Type} &
{\bfseries Description}\\}
\tablehead{\hline
{\bfseries Property} &
{\bfseries Type} &
{\bfseries Description}\\}
\tabletail{}
\tablelasttail{}
\begin{supertabular}{|m{3.665cm}|m{1.278cm}|m{12.052cm}|}
\hline
{\itshape Length} &
double &
Length of the region in which the field is present.\\\hline
{\itshape CavityMode} &
string &
Phasing mode of the cavity - options are Phased, Unphased and Electrostatic.\\\hline
{\itshape FieldDuringPhasing} &
string &
Controls the field during cavity phasing -- options are None, Electrostatic, TimeVarying and
EnergyGainOptimised.\\\hline
{\itshape EnergyGain} &
double &
WhenFieldDuringPhasing is set to EnergyGainOptimised, controls the peak electric field.\\\hline
{\itshape Frequency} &
double &
The cavity frequency.\\\hline
{\itshape PeakEField} &
double &
The peak field of the cavity. Not used when the FieldDuringPhasing is EnergyGainOptimised.\\\hline
{\itshape TimeDelay} &
double &
In Phased mode the time delay (absolute time) of the cavity.\\\hline
PhasingVolume &
string &
Set to SpecialVirtual to make the central volume a special virtual.\\\hline
\multicolumn{2}{m{5.143cm}|}{\hspace*{-\tabcolsep}\begin{tabular}{|m{3.665cm}|m{1.278cm}}

ReferenceParticle

Energy &
double\\\hline
ReferenceParticle

Charge &
double\\\hline
\end{tabular}\hspace*{-\tabcolsep}
} &
In Electrostatic mode, G4MICE calculates the peak field and the field the reference particle sees using a combination of
the reference particle energy, charge and phase. Take defaults from datacards NominalKineticEnergy and MuonCharge
\\\hhline{~~-}
ReferenceParticle

Phase &
double &
G4MICE tries to phase the field so that the reference particle crosses the cavity at ReferenceParticlePhase (units are
angular). 0\textsuperscript{o} corresponds to no energy gain, 90\textsuperscript{o} corresponds to operation on-crest.
Default from datacard rfAcclerationPhase.\\\hline
\end{supertabular}
\end{center}
\subsection{FieldType RFFieldMap}
Sets a cavity with an RF field map in a particular region. RFFieldMap uses the same phasing algorithm as described
above.

\begin{center}
\tablefirsthead{\hline
{\bfseries Property} &
{\bfseries Type} &
{\bfseries Description}\\}
\tablehead{\hline
{\bfseries Property} &
{\bfseries Type} &
{\bfseries Description}\\}
\tabletail{}
\tablelasttail{}
\begin{supertabular}{|m{3.665cm}|m{1.278cm}|m{12.052cm}|}
\hline
{\itshape Length} &
double &
Length of the region in which the field is present.\\\hline
{\itshape CavityMode} &
string &
Phasing mode of the cavity - options are Phased and Unphased. RFFieldMaps cannot operated in Electrostatic mode.\\\hline
{\itshape FieldDuringPhasing} &
string &
Controls the field during cavity phasing -- options are None, Electrostatic, TimeVarying and
EnergyGainOptimised.\\\hline
{\itshape EnergyGain} &
double &
WhenFieldDuringPhasing is set to EnergyGainOptimised, controls the peak electric field.\\\hline
{\itshape Frequency} &
double &
The cavity frequency.\\\hline
{\itshape PeakEField} &
double &
The peak field of the cavity. Not used when the FieldDuringPhasing is EnergyGainOptimised.\\\hline
{\itshape TimeDelay} &
double &
In Phased mode the time delay (absolute time) of the cavity.\\\hline
PhasingVolume &
string &
Set to SpecialVirtual to make the central volume a special virtual.\\\hline
\multicolumn{2}{m{5.143cm}|}{\hspace*{-\tabcolsep}\begin{tabular}{|m{3.665cm}|m{1.278cm}}

ReferenceParticle

Energy &
double\\\hline
ReferenceParticle

Charge &
double\\\hline
\end{tabular}\hspace*{-\tabcolsep}
} &
In Electrostatic mode, G4MICE calculates the peak. field and the field the reference particle sees using a combination
of the reference particle energy, charge and phase. Take defaults from datacards NominalKineticEnergy and MuonCharge
\\\hhline{~~-}
ReferenceParticle

Phase &
double &
G4MICE tries to phase the field so that the reference particle crosses the cavity at ReferenceParticlePhase (units are
angular). 0\textsuperscript{o} corresponds to no energy gain, 90\textsuperscript{o} corresponds to operation on-crest.
Default from datacard rfAcclerationPhase.\\\hline
{\itshape FileName} &
string &
The file name of the field map file.\\\hline
{\itshape FileType} &
string &
The file type of the field map. Only supported option is SuperFishSF7.\\\hline
\end{supertabular}
\end{center}
\subsection{}
\clearpage\subsection{FieldType Multipole}
Creates a multipole of arbitrary order. Fields are generated using either a hard edged model, with no fringe fields at
all; or an Enge model similar to ZGoubi and COSY. In the former case fields are calculated using a simple polynomial
expansion. In the latter case fields are calculated using the polynomial expansion with an additional exponential drop
off. Fields can be superimposed onto a bent coordinate system to generate a sector multipole with arbitrary fixed
radius of curvature.

Unlike most other field models in G4MICE, the zero position corresponds to the center of the entrance of the multipole;
and the multipole extends in the +z direction.

The method to define end fields is described in the section EndFieldTypes below

\begin{center}
\tablefirsthead{\hline
{\bfseries Property} &
{\bfseries Type} &
{\bfseries Description}\\}
\tablehead{\hline
{\bfseries Property} &
{\bfseries Type} &
{\bfseries Description}\\}
\tabletail{}
\tablelasttail{}
\begin{supertabular}{|m{3.691cm}|m{1.252cm}|m{12.052cm}|}
\hline
{\itshape Pole} &
int &
The reference pole of the magnet. 1=dipole, 2=quadrupole, 3=sextupole etc.\\\hline
{\itshape FieldStrength} &
double &
Scale the field strength in the good field region. For dipoles, this sets the dipole field; for quadrupoles this sets
the field gradient. Note that for some end field settings there can be no good field region (e.g. if the end length is
{\textgreater}\~{} centre length).\\\hline
{\itshape Height} &
double &
Height of the field region.\\\hline
{\itshape Width} &
double &
Width or delta radius of the field region.\\\hline
{\itshape Length} &
double &
Length of the field along the bent trajectory.\\\hline
EndFieldType &
string &
Set to HardEdged to disable fringe fields. Set to Enge or Tanh to use those models, as described elsewhere. Default is
HardEdged.\\\hline
CurvatureModel &
string &
Choose the model for curvature. Straight forces no curvature. Constant gives a constant radius of curvature;
StraightEnds gives a constant radius of curvature along the length of the multipole with straight end fields beyond
this length. MomentumBased gives radius of curvature determined by a momentum and a total bending angle.\\\hline
ReferenceCurvature &
double &
Radius of curvature of the magnet in Constant or StraightEnds mode. Set to 0 for a straight magnet. Default is
0.\\\hline
ReferenceMomentum &
double &
Reference momentum used to calculate the radius of curvature of a dipole in MomentumBased mode. Default is 0.\\\hline
{\itshape BendingAngle} &
double &
The angle used to calculate the radius of curvature of a dipole in MomentumBased mode. Note that this is mandatory in
MomentumBased mode.\\\hline
\end{supertabular}
\end{center}

\bigskip

\clearpage\subsection{FieldType CombinedFunction}
This creates superimposed dipole, quadrupole and sextupole fields with a common radius of curvature. The field is
intended to mimic the first few terms in a multipole expansion of a field like

\begin{equation*}
B(y=0)=B_0\left(\frac r{r_0}\right)^k
\end{equation*}
The field index is a user defined parameter, while the dipole field and radius of curvature can either be defined
directly by the user or defined in terms of a reference momentum and total bending angle. Fields are calculated as in
the multipole field type defined above.

\begin{center}
\tablefirsthead{\hline
{\bfseries Property} &
{\bfseries Type} &
{\bfseries Description}\\}
\tablehead{\hline
{\bfseries Property} &
{\bfseries Type} &
{\bfseries Description}\\}
\tabletail{}
\tablelasttail{}
\begin{supertabular}{|m{3.689cm}|m{1.252cm}|m{12.054cm}|}
\hline
{\itshape Pole} &
int &
The reference pole of the magnet. 1=dipole, 2=quadrupole, 3=sextupole etc.\\\hline
{\itshape BendingField} &
double &
The nominal dipole field \textit{B}\textit{\textsubscript{0}}. Note that this is mandatory in all cases except where
CurvatureModel is MomentumBased, when the BendingAngle and ReferenceMomentum is used to calculate the \ dipole field
instead.\\\hline
{\itshape FieldIndex} &
double &
The field index \textit{k}.\\\hline
{\itshape Height} &
double &
Height of the field region.\\\hline
{\itshape Width} &
double &
Width or delta radius of the field region.\\\hline
{\itshape Length} &
double &
Length of the field along the bent trajectory.\\\hline
EndFieldType &
string &
Set to HardEdged to disable fringe fields. Set to Enge or Tanh to use those models, as described elsewhere. Default is
HardEdged.\\\hline
CurvatureModel &
string &
Choose the model for curvature. Straight forces no curvature. Constant gives a constant radius of curvature;
StraightEnds gives a constant radius of curvature along the length of the multipole with straight end fields beyond
this length. MomentumBased gives radius of curvature determined by a momentum and a total bending angle.\\\hline
ReferenceCurvature &
double &
Radius of curvature of the magnet in Constant or StraightEnds mode. Set to 0 for a straight magnet. Default is
0.\\\hline
ReferenceMomentum &
double &
Reference momentum used to calculate the radius of curvature of a dipole in MomentumBased mode. Default is 0.\\\hline
{\itshape BendingAngle} &
double &
The angle used to calculate the radius of curvature of a dipole in MomentumBased mode. Note that this is mandatory in
MomentumBased mode.\\\hline
\end{supertabular}
\end{center}
\clearpage\subsection{EndFieldTypes}
In the absence of current sources, the magnetic field can be calculated from a scalar potential using the standard
relation

\begin{equation*}
\vec B=\nabla V_n
\end{equation*}
The scalar magnetic potential of the n\textsuperscript{th}{}-order multipole field is given by

\begin{equation*}
V_n=\sum _{q=0}^{q_m}\sum _{m=0}^nn!^2\frac{G^{(2q)}(s)(r^2+y^2)^q\sin (\frac{m\pi } 2)r^{n-m}y^m}{4^qq!(n+q)!m!(n-m)!}
\end{equation*}
where \textit{G(s)} is either the double Enge function,

\begin{equation*}
G(s)=E[(x-x_0)/\lambda ]+E[(-x-x_0)/\lambda ]-1
\end{equation*}
\begin{equation*}
E(s)=\frac{B_0}{R_0^n}\frac 1{1+\exp (C_1+C_2s+C_3s^2+...)}
\end{equation*}
or G(s) is the double tanh function,

\begin{equation*}
G(s)=\tanh [(x+x_0)/\lambda ]/2+\tanh [(x-x_0)/\lambda ]/2
\end{equation*}
and \textit{(r, y, s)} is the position vector in the rotating coordinate system. Note that here s is the distance from
the nominal end of the field map.

\begin{center}
\tablefirsthead{\hline
{\bfseries Property} &
{\bfseries Type} &
{\bfseries Description}\\}
\tablehead{\hline
{\bfseries Property} &
{\bfseries Type} &
{\bfseries Description}\\}
\tabletail{}
\tablelasttail{}
\begin{supertabular}{|m{3.689cm}|m{1.252cm}|m{12.054cm}|}
\hline
EndFieldType &
string &
Set to HardEdged to disable fringe fields. Set to Enge or Tanh to use those models, as described elsewhere. Default is
HardEdged.\\\hline
\multicolumn{3}{|m{17.394999cm}|}{{\itshape The following properties are used for EndFieldType Tanh}}\\\hline
EndLength &
double &
Set the l parameter that defines the rapidity of the field fall off.\\\hline
CentreLength &
double &
Set the \textit{x}\textit{\textsubscript{0}} parameter that defines the length of the flat field region.\\\hline
MaxEndPole &
int &
Set the maximum pole that will be calculated -- should be larger than the multipole pole.\\\hline
\multicolumn{3}{|m{17.394999cm}|}{{\itshape The following properties are used for EndFieldType Enge}}\\\hline
EndLength &
double &
Set the l parameter that defines the rapidity of the field fall off.\\\hline
CentreLength &
double &
Set the \textit{x}\textit{\textsubscript{0}} parameter that defines the length of the flat field region.\\\hline
MaxEndPole &
int &
Set the maximum pole that will be calculated -- should be larger than the multipole pole.\\\hline
\multicolumn{2}{m{5.1410003cm}|}{\hspace*{-\tabcolsep}\begin{tabular}{|m{3.689cm}|m{1.252cm}}

Enge1 &
double\\\hline
Enge2 &
double\\\hline
... &
double\\\hline
EngeN &
double\\\hline
\end{tabular}\hspace*{-\tabcolsep}
} &
Set the parameters C\textsubscript{i} as defined in the Enge function above.\\\hhline{~~-}
\end{supertabular}
\end{center}
\subsection{FieldType MagneticFieldMap}
Reads or writes a magnetic field map in a particular region. Two sorts of field maps are supported; either a 2d field
map, in which cylindrical symmetry is assumed, or a 3d field map. 

For 2d field maps, G4MICE reads or writes a file that contains information about the radial and longitudinal field
components. This is intended for solenoidal field maps where only radial and longitudinal field components are present.
Note that in write mode, G4MICE assumes cylindrical symmetry of the fields. In this case, G4MICE writes the \textit{x}
and \textit{z} components of the magnetic field at points on a grid in \textit{x} and \textit{z}. Fields with an
electric component are excluded from this summation.

For 3d field maps, G4MICE reads a file that contains the position and field in cartesian coordinates and performs a
linear interpolation. This requires quite large field map files; the file size can be slightly reduced by using certain
symmetries, as described below. It is currently not possible to write 3d field maps.

\begin{center}
\tablefirsthead{\hline
{\bfseries Property} &
{\bfseries Type} &
{\bfseries Description}\\}
\tablehead{\hline
{\bfseries Property} &
{\bfseries Type} &
{\bfseries Description}\\}
\tabletail{}
\tablelasttail{}
\begin{supertabular}{|m{3.665cm}|m{1.278cm}|m{12.052cm}|}
\hline
{\itshape FieldMapMode} &
string &
Set to Read to read a field map; and Write to write a field map.\\\hline
{\itshape FileName} &
string &
The file name that is used for reading or writing.\\\hline
FileType &
string &
The file format. Supported options in Read mode are g4micetext, g4micebinary, g4beamline, icool, g4bl3dGrid. Only
g4micetext is supported in Write mode. Default is g4micetext.\\\hline
Symmetry &
string &
Symmetry option for g4bl3dGrid file type. Options are None, Dipole or Quadrupole. None uses the field map as is, while
Dipole and Quadrupole reflect the octant between the positive \textit{x}, \textit{y} and \textit{z} axes to give an
appropriate field for a dipole or quadrupole.\\\hline
\multicolumn{2}{m{5.143cm}|}{\hspace*{-\tabcolsep}\begin{tabular}{|m{3.665cm}|m{1.278cm}}

{\itshape ZStep} &
double\\\hline
{\itshape RStep} &
double\\\hline
\end{tabular}\hspace*{-\tabcolsep}
} &
Step size in \textit{z} and \textit{r}. Mandatory in Write mode but not used in Read mode (where step size comes from
the map file).\\\hhline{~~-}
\multicolumn{2}{m{5.143cm}|}{\hspace*{-\tabcolsep}\begin{tabular}{|m{3.665cm}|m{1.278cm}}

{\itshape ZMin} &
double\\\hline
{\itshape ZMax} &
double\\\hline
{\itshape RMin} &
double\\\hline
{\itshape RMax} &
double\\\hline
\end{tabular}\hspace*{-\tabcolsep}
} &
Upper and lower bounds in \textit{z} and \textit{r}. Mandatory in Write mode but not used in Read mode (where boundaries
come from the map file).\\\hhline{~~-}
\end{supertabular}
\end{center}
Some file formats are described below. I am working towards making the file format more generic and hence possibly
easier to use, but backwards compatibility will hopefully be maintained. 

\clearpage\subsubsection{g4micetext Field Map Format}
The native field map format used by g4mice in text mode is described below.

{\ttfamily
\# GridType = Uniform N = \textbf{number\_rows}}

{\ttfamily
\# Z1 = \textbf{z\_start} Z2 = \textbf{z\_end} dZ = \textbf{z\_step}}

{\ttfamily
\# R1 = \textbf{r\_start} R2 = \textbf{r\_end} dR = \textbf{r\_step}}

{\ttfamily\bfseries
Bz\_Values\ \ Br\_Values}

{\ttfamily\bfseries
...\ \ \ \ ...}

{\ttfamily\bfseries
{\textless}Repeat as necessary{\textgreater}}

In this mode, field maps are represented by field values on a regular 2d grid that is assumed to have solenoidal
symmetry, i.e. cylindrical symmetry with no tangential component.

\begin{center}
\tablefirsthead{\hline
{\bfseries Name} &
{\bfseries Type} &
{\bfseries Description}\\}
\tablehead{\hline
{\bfseries Name} &
{\bfseries Type} &
{\bfseries Description}\\}
\tabletail{}
\tablelasttail{}
\begin{supertabular}{|m{3.665cm}|m{2.305cm}|m{11.025001cm}|}
\hline
{\ttfamily\bfseries number\_rows} &
double &
Number of rows in the field map file.\\\hline
{\ttfamily\bfseries z\_start} &
double &
Position of the grid start along the \textit{z} axis.\\\hline
{\ttfamily\bfseries z\_end} &
double &
Position of the grid end along the \textit{z} axis.\\\hline
{\ttfamily\bfseries z\_step} &
double &
Step size in \textit{z}.\\\hline
{\ttfamily\bfseries r\_start} &
double &
Position of the grid start along the \textit{r} axis.\\\hline
{\ttfamily\bfseries r\_end} &
double &
Position of the grid end along the \textit{r} axis.\\\hline
{\ttfamily\bfseries r\_step} &
double &
Step size in \textit{r}.\\\hline
{\ttfamily\bfseries Bz\_Values} &
double &
\textit{Bz} field value.\\\hline
{\ttfamily\bfseries Br\_Values} &
double &
\textit{Br} field value.\\\hline
\end{supertabular}
\end{center}
\subsubsection{g4bl3dGrid Field Map Format}
The file format for 3d field maps is a slightly massaged version of a file format used by another code, g4beamline. In
this mode, field maps are represented by field values on a regular cartesian 3d grid.

{\ttfamily\bfseries
number\_x\_points number\_y\_points number\_z\_points global\_scale}

{\ttfamily
1 X [\textbf{x\_scale}]}

{\ttfamily
2 Y [\textbf{y\_scale}]}

{\ttfamily
3 Z [\textbf{z\_scale}]}

{\ttfamily
4 BX [\textbf{bx\_scale}]}

{\ttfamily
5 BY [\textbf{by\_scale}]}

{\ttfamily
6 BZ [\textbf{bz\_scale}]}

{\ttfamily
0}

{\ttfamily\bfseries
X\_Values \ \ Y\_Values\ \ Z\_Values\ \ Bx\_values\ \ By\_values\ \ Bz\_values}

{\ttfamily\bfseries
...\ \ \ \ ...\ \ \ \ ...\ \ \ \ ...\ \ \ \ ...\ \ \ \ ...}

{\ttfamily\bfseries
{\textless}Repeat as necessary{\textgreater}}

where text in bold indicates a value described in the following table

\begin{center}
\tablefirsthead{\hline
{\bfseries Name} &
{\bfseries Type} &
{\bfseries Description}\\}
\tablehead{\hline
{\bfseries Name} &
{\bfseries Type} &
{\bfseries Description}\\}
\tabletail{}
\tablelasttail{}
\begin{supertabular}{|m{3.665cm}|m{2.305cm}|m{11.025001cm}|}
\hline
{\ttfamily\bfseries number\_x\_points} &
double &
Number of points along x axis.\\\hline
{\ttfamily\bfseries number\_y\_points} &
double &
Number of points along y axis.\\\hline
{\ttfamily\bfseries number\_z\_points} &
double &
Number of points along z axis.\\\hline
{\ttfamily\bfseries global\_scale} &
double &
Global scale factor applied to all x, y, z and Bx, By, Bz values.\\\hline
{\ttfamily\bfseries x\_scale} &
double &
Scale factor applied to all x values.\\\hline
{\ttfamily\bfseries y\_scale} &
double &
Scale factor applied to all y values.\\\hline
{\ttfamily\bfseries z\_scale} &
double &
Scale factor applied to all z values.\\\hline
{\ttfamily\bfseries bx\_scale} &
double &
Scale factor applied to all Bx values.\\\hline
{\ttfamily\bfseries by\_scale} &
double &
Scale factor applied to all By values.\\\hline
{\ttfamily\bfseries bz\_scale} &
double &
Scale factor applied to all Bz values.\\\hline
{\ttfamily\bfseries X\_Values} &
double &
List (column) of each x value.\\\hline
{\ttfamily\bfseries Y\_Values} &
double &
List (column) of each y value.\\\hline
{\ttfamily\bfseries Z\_Values} &
double &
List (column) of each z value.\\\hline
{\ttfamily\bfseries Bx\_Values} &
double &
List (column) of each Bx value.\\\hline
{\ttfamily\bfseries By\_Values} &
double &
List (column) of each By value.\\\hline
{\ttfamily\bfseries Bz\_Values} &
double &
List (column) of each Bz value.\\\hline
\end{supertabular}
\end{center}

\bigskip
\end{document}
