\input{rgb}

\lstset{% general command to set parameter(s)
frame=single,
basicstyle=\small,
keywordstyle=\color{DarkViolet}\bfseries,
identifierstyle=\color{DarkGreen},
commentstyle=\color{DarkRed},
stringstyle=\ttfamily,% typewriter type for strings
showstringspaces=false,% no special string spaces
captionpos=b% caption at bottom
}

\chapter{Introduction to the MAUS API}
\label{chapter:api}
This chapter introduces the MAUS API framework and looks in depth at the
structure of the classes and interfaces that it comprises of. Several example
\emph{minimal} implementations are given before a note on scalability and
extending the framework.

\section{Motivation}
The MAUS API framework provides MAUS developers with a well-defined environment
for developing reconstruction code, while allowing independent development of
the backend and code sharing of common elements like error handling and data
mangling.

\section{Everything starts with a `Module'}
An \emph{Module} is the basic building block of the MAUS API framework. Four
types of module exist within MAUS.

\begin{itemize}
\item An Input module is used to create an instance of a MAUS Spill structure.
\item An Output module is used to store an instance of a MAUS data structure.
\item A Map module is used to modify a single Spill item. This enables the
reconstruction to be parallelised across multiple Maps.
\item A Reduce module is used to act on a collection of Spills.
\end{itemize}

Every module has a constructor, destructor, a birth and a death method. Input
modules have an emitter function that yields a new data object. Output modules 
have a save function that takes a data object and stores it (either on disk, or
for example broadcasting across a socket).

Map modules and Reduce modules have a process function that takes a data object 
and modifies it. The important difference is that Map modules have no internal
state, meaning that they can be run in parallel. Reduce modules, on the other
hand, do have internal state. This means that they can act on groups of Spills,
for example collecting histogram data.

\section{Inheritance}
In order to correctly make a module, one should inherit from the correct type.

\begin{itemize}
\item Input modules should inherit from InputBase.
\item Output modules should inherit from OutputBase.
\item Map modules should inherit from MapBase.
\item Reduce modules should inherit from ReduceBase.
\end{itemize}

Base types are defined in \verb|src/common_cpp/API| for C++ modules and
\verb|src/common_py/API| for python modules.

\section{Data Mangling}
MAUS supports representation of the data structure in various different formats.
MAUS support representation in ROOT, ascii string and json formats. It is
recommended that reconstruction routines use the ROOT format. For legacy
reasons, MAUS supports reconstruction of data stored in ascii or json formats.

In python, the representation (i.e. format) of the data can be changed by using 
the module \verb|maus_cpp.converter|. If a module implements conversion to a
specific data type, the \verb|can_convert| flag should be set to True; otherwise 
MAUS will always hand data in string format.

In C++, mappers are templated to a MAUS data type. The API then handles any
necessary conversion to that data type, and provides the appropriate python
wrapper code for that module.

Currently only map modules support data mangling.

\section{Module Initialisation and Destruction}
MAUS has two execution concepts. A \verb|Job| refers to a single execution of the code, while a \verb|Run| refers to the processing of data for a MICE data run or Monte Carlo run.

In MAUS, Inputters, Mappers, Reducers and Outputters are initialised at the start of every Job and destructed at the end of every Job. \verb|birth(...)| for Inputters and Outputters is called at the start of every Job and \verb|death()| is called at the end of every Job. The \verb|birth(...)| for Mappers and Reducers is called at the start of every Run and \verb|death()| is called at the end of every Run.

The logic is that for each code execution we typically want to access data from a single data source and output data to a single data file. But mappers and reducers are reinitialised for each run to enable loading of new calibrations, etc. It is required that all transient information about the reconstruction pertaining to a run - particularly ID of the calibration and cabling used - is recorded in the StartOfRun data structure. Any summary information on code execution during the run may be stored in the EndOfRun data structure. All transient information pertaining to a job - for example code version or bzr branch - should be recorded in the StartOfJob data structure. Any summary information on code execution during the job may be stored in the EndOfJob data structure.

\section{Global Objects - Objects for Many Modules}
There are some objects that sit outside the scope of the modular framework described above. Typically these are objects that do not belong to any one module, but need to be accessed by many. Examples are the logging functionality (Squeak), ErrorHandler, Configuration datacards, field maps, geometry description and Geant4 interfaces. These are accessed through the static singleton class \verb|Globals| defined in \verb|src/common_cpp/Utils/Globals.hh|. Initialisation is handled in \verb|src/common_cpp/Globals/GlobalsManager.hh|. One Globals instance is initialised per subprocess when running in multiprocessing mode.

For python users, some Global objects can be accessed by reference to the \verb|maus_cpp.globals module|.

\subsection{Global Object Initialisation}
Global objects are initialised before any modules in Go.py and deleted after all modules are deathed. Global object initialisation and destruction is handled at the Job level by \verb|src/common_cpp/Globals/GlobalManager.hh| and called in python via \verb|maus_cpp.globals| as above.

Run-by-run initialisation is handled by the RunActionManager, defined in\linebreak\verb|src/common_cpp/Utils/RunActionManager.hh|. The RunActionManager holds a list of objects inheriting from \verb|RunActionBase| each of which defines functions to call at the start and end of each run.

\section{Internal Classes}
The following classes and namespaces are used to provide an interface between
reconstruction modules and the framework (backend).

\subsection{Abstraction Layers}
These are all defined in \verb|src/common_cpp/API| and \verb|src/common_py/API|
folders

\begin{itemize}
\item IModule - interface class for all modules; defines \verb|birth| and 
\verb|death|
\item ModuleBase - base class for modules, includes some error handling.
\item IInput - interface class for all inputs; defines \verb|emitter| and 
              inherits from IModule
\item InputBase - base class for all inputs, includes some error handling and
             inherits from ModuleBase and IInput
\item IMap - interface class for all inputs; defines \verb|process| and 
              inherits from IModule
\item MapBase - base class for all inputs, includes some error handling and 
              inherits from ModuleBase and IMap
\item IReduce - interface class for all reducers; defines \verb|process| and 
              inherits from IModule
\item ReduceBase - base class for all reducers, includes some error handling and
              inherits from ModuleBase and IReduce
\item IOutput - interface class for all outputs; defines \verb|save| and 
              inherits from IModule
\item OutputBase - base class for all outputs, includes some error handling and 
              inherits from ModuleBase and IOutput
\end{itemize}

\subsection{C++ Python Wrapper}
\verb|src/common_cpp/API/PyWrapMapBase| is a templated class that wraps a 
generic map object and provides python interfaces to that map object.

Currently, Input, Reduce, Output wrappers are provided by SWIG.

\subsection{Data Mangling}
Data mangling is handled in a variety of layers.
\begin{itemize}
\item \verb|src/common_cpp/Converter/ConverterBase| provides an abstraction for
      conversion from one type to another
\item \verb|src/common_cpp/Converter/DataConverters| provides implementations
      of the data conversions
\item \verb|src/common_cpp/Converter/ConverterFactory| provides a function like\linebreak
      \verb|TYPE2* convert<TYPE1, TYPE2>(TYPE1* data)| with implementations for
      each of the types. This then provides explicit conversion (i.e. where both
      input and output types are known.
\item \verb|src/common_cpp/Utils/PyObjectWrapper| provides functions for 
      wrapping all of the data types into a \verb|PyObject*|. It also provides a 
      function that unwraps the \verb|PyObject*|, figures out the data type and 
      returns a data of the appropriate type.
\item \verb|src/common_cpp/API/PyWrapMapBase| calls unwrap based on the type
      stored in the \verb|PyObject*|.
\end{itemize}
